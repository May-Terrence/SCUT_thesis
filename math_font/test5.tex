\documentclass{article}
\usepackage{xcolor}  
\usepackage{enumitem}  
\usepackage[no-math]{fontspec}        
\usepackage{xeCJK}            
\usepackage{booktabs}   
\usepackage{amsmath}   
\usepackage{latexsym}  
\usepackage{yhmath}    
\usepackage{amssymb}  
\usepackage{eucal}   % \EuScript
\usepackage{mathrsfs}    
% 设置正文字体
\setmainfont{Times New Roman} % 正文数字和字母使用 Times New Roman
% 设置中文正文字体为 SimSun
\setCJKmainfont{SimSun}       % 正文中文使用 SimSun
\usepackage{unicode-math} 
\setmathfont{XITS Math}[math-style=ISO,bold-style=ISO]%注意math-style设定不同符号默认的直体斜体,bold-style设定加粗后的行为(有可能斜体改直体后加粗,取决于具体命令)
\setmathfont{XITS Math}[range={cal, bfcal}, StylisticSet=1]

%%%%%%%%%%%%下面的定义是lshort书中符号表所需要的定义%%%%%%%%%%%%%%%%%%%%%%%%%%%%%%%
\newenvironment{symbols}[1]%
  {\small\def\arraystretch{1.5
  }
  \begin{tabular}{@{}#1@{}}}%
  {\end{tabular}}
%%%%%%%%%%%%%%%%%%%%%%%%%%%%%%%%%%%%%%%%%%%%%%%%%%%%%%%%%%%%%%%%%%%%%%%%%%%%%%%%%%%%%%%%
\begin{document}
%%%%%%%%%%%%%%%%%%%%%%%%%%%%%%%
这是正文部分,中文使用 SimSun 字体,英文使用 Times New Roman 字体。

数学公式字符规范说明
% $\mathsf{ABCDEFG} $,$\symsf{ABCDEFG} $,$\symsfup{ABCDEFG} $,$\symsfit{ABCDEFG} $

1. 字体样式

1.1 普通变量

数学公式中的普通变量使用斜体 (italic)

普通变量: $x, y, z, f(x)$

1.2 希腊字母

希腊字母默认使用斜体,除非具有特定含义

希腊字母: $\alpha, \beta, \Gamma$ (斜体)

特定常量: $\pi, \Delta$ (直体)
\begin{equation}
  \alpha, \beta, \gamma, \delta, \varepsilon, \epsilon, \zeta, \eta, \theta, \vartheta, \iota, \kappa, \lambda, \mu, \nu, \xi, \pi ,\varpi, \rho, \varrho, \sigma, \varsigma, \tau, \upsilon, \varphi, \phi, \chi, \psi, \omega
\end{equation}
\begin{equation}
    \symup{\alpha, \beta, \gamma, \delta, \varepsilon, \epsilon, \zeta, \eta, \theta, \vartheta, \iota, \kappa, \lambda, \mu, \nu, \xi, \pi ,\varpi, \rho, \varrho, \sigma, \varsigma, \tau, \upsilon, \varphi, \phi, \chi, \psi, \omega}
\end{equation}
\begin{equation}
  \Gamma, \Delta, \Theta, \Lambda, \Xi, \Pi, \Sigma, \Upsilon, \Phi, \Psi, \Omega
\end{equation}
\begin{equation}
  \symup{\Gamma, \Delta, \Theta, \Lambda, \Xi, \Pi, \Sigma, \Upsilon, \Phi, \Psi, \Omega}
\end{equation}
1.3 数学函数

数学函数(如三角函数、对数等)使用直体 (upright)

数学函数: $\sin, \cos, \log, \exp$

1.4 矢量与矩阵

矢量和矩阵通常使用粗体 (boldface)

矢量: $\symbf{v}, \boldsymbol{u}$,$\mathbf{v}, \mathbf{u}$
\begin{equation}
  \boldsymbol{\alpha, \beta, \gamma, \delta, \varepsilon, \epsilon, \zeta, \eta, \theta, \vartheta, \iota, \kappa, \lambda, \mu, \nu, \xi, \pi ,\varpi, \rho, \varrho, \sigma, \varsigma, \tau, \upsilon, \varphi, \phi, \chi, \psi, \omega}
\end{equation}
\begin{equation}
  \symbf{\alpha, \beta, \gamma, \delta, \varepsilon, \epsilon, \zeta, \eta, \theta, \vartheta, \iota, \kappa, \lambda, \mu, \nu, \xi, \pi ,\varpi, \rho, \varrho, \sigma, \varsigma, \tau, \upsilon, \varphi, \phi, \chi, \psi, \omega}
\end{equation}

矩阵: $\boldsymbol{A}, \boldsymbol{B}$ 或 $\mathcal{A}, \mathcal{B}$
\begin{equation}
  \boldsymbol{\Gamma, \Delta, \Theta, \Lambda, \Xi, \Pi, \Sigma, \Upsilon, \Phi, \Psi, \Omega}
\end{equation}
\begin{equation}
  \symbf{\Gamma, \Delta, \Theta, \Lambda, \Xi, \Pi, \Sigma, \Upsilon, \Phi, \Psi, \Omega}
\end{equation}

英文字母小写:
\begin{equation}
  abcdefghijklmnopqrstuvwxyz 
\end{equation}
\begin{equation}
  \boldsymbol{abcdefghijklmnopqrstuvwxyz} 
\end{equation}
\begin{equation}
  \mathrm{abcdefghijklmnopqrstuvwxyz}
\end{equation}
\begin{equation}
  \mathbf{abcdefghijklmnopqrstuvwxyz}
\end{equation}

英文字母大写:
\begin{equation}
  ABCDEFGHIJKLMNOPQRSTUVWXYZ 
\end{equation}
\begin{equation}
  \boldsymbol{ABCDEFGHIJKLMNOPQRSTUVWXYZ} 
\end{equation}
\begin{equation}
  \mathrm{ABCDEFGHIJKLMNOPQRSTUVWXYZ}
\end{equation}
\begin{equation}
  \mathbf{ABCDEFGHIJKLMNOPQRSTUVWXYZ}
\end{equation}
1.5 特殊集合与符号

常用的数域和集合使用黑板体 (blackboard bold)

集合: $\mathbb{R}, \mathbb{Z}, \mathbb{N}$

1.6 特殊结构

使用花体 (calligraphic) 和分数体 (fraktur) 表示特定结构

花体: $\mathcal{F}, \mathcal{L}$

分数体: $\mathfrak{g}, \mathfrak{h}$


2. 上下标的样式规范

2.1 普通上下标

如果上下标是变量,使用斜体 (italic);如果是固定值或标记,使用直体 (upright)

变量下标: $x_i, A_{ij}, v_{n+1}$ (斜体)

固定下标: $R_\mathrm{earth}, E_\text{total}$ (直体)

加权矩阵$\symbf{A}_{i}$和 ${{\boldsymbol{W}}_{2}}$ 是对称矩阵,且$ {{\boldsymbol{W}}_{2}}$非奇异。	 

2.2 混合上标与下标

上标和下标可混合使用,注意字体语义的一致性

混合上下标: $x_i^2, A_{ij}^k, E_\mathrm{total}^2$

2.3 特殊集合的上下标

集合与空间的上下标可以结合黑板体和花体

特殊上下标: $\mathbb{R}_n, \mathcal{F}_t$


3. LaTeX 排版示例

3.1 定义数学字体

调用 amsmath 和 amssymb 宏包

3.2 使用具体字体

普通变量: $x, y, z$
数学函数: $\sin, \cos, \log$
矢量与矩阵: $\boldsymbol{v}, \boldsymbol{A}$
集合与符号: $\mathbb{R}, \mathcal{F}, \mathfrak{g}$
固定上下标: $R_\mathrm{earth}, E_\text{total}$




% ### **方法:正确加粗斜体希腊字母**
% 使用 `unicode-math` 时,需区分默认的斜体和直体样式。为了加粗希腊字母并保持斜体,需要确保字体和命令的正确配置。

% #### **完整代码示例**
% 默认斜体希腊字母
默认斜体希腊字母:$\alpha, \beta, \gamma, \pi$

% 加粗斜体希腊字母
加粗斜体希腊字母:$\boldsymbol{\alpha}, \boldsymbol{\beta}, \boldsymbol{\gamma}, \boldsymbol{\pi}$

% 默认直体希腊字母
默认直体希腊字母:$\symup{\alpha}, \symup{\beta}, \symup{\gamma}, \symup{\pi}$

% 加粗直体希腊字母
加粗直体希腊字母:$\boldsymbol{\symup{\alpha}}, \boldsymbol{\symup{\beta}}, \boldsymbol{\symup{\gamma}}, \boldsymbol{\symup{\pi}}$

加粗效果:$\symbf{{\alpha}}, \symbf{{\beta}}, \symbf{\symup{\gamma}}, \symbf{\symup{\pi}}$. $A$,$\symbf{A}$, 

${\mathbf{\gamma}}$, ${\mathbf{\pi}}$. %不起作用

$\symbf{{\alpha}}, \symbf{{\beta}}, \symbf{\symup{\gamma}}, \symbf{\symup{\pi}}$.

下表为lshort中文版中,表 4.2: 数学字母字体的内容,新增了一行测试mathcal命令。

\begin{table}[htp]
  \centering
  \caption{数学字母字体} \label{tbl:math-fonts}
  \begin{tabular}{*{3}{l}}
  \hline
  \textbf{示例}    & \textbf{命令} & \textbf{依赖的宏包}\\
  \hline
  $\mathnormal{ABCDE abcde 1234}$  & {mathnormal}\{\ldots\}&       \\
  $\mathrm{ABCDE abcde 1234}$      & {mathrm}\{\ldots\}    &       \\
  $\mathit{ABCDE abcde 1234}$      & {mathit}\{\ldots\}    &       \\
  $\mathbf{ABCDE abcde 1234}$      & {mathbf}\{\ldots\}    &       \\
  $\mathsf{ABCDE abcde 1234}$      & {mathsf}\{\ldots\}    &       \\
  $\mathtt{ABCDE abcde 1234}$      & {mathtt}\{\ldots\}    &       \\
  $\mathcal{ABCDE}$                  & {mathcal}\{\ldots\}   &     \\ 
  $\CMcal{ABCDE}$                  & {CMcal}\{\ldots\}   & 仅提供大写字母 \\
  \hline
  $\EuScript{ABCDE}$               & {mathcal}\{\ldots\}   & {eucal} 仅提供大写字母 \\
  $\mathscr{ABCDE}$                & {mathscr}\{\ldots\}   & {mathrsfs} 仅提供大写字母\\
  $\mathfrak{ABCDE abcde 1234}$    & {mathfrak}\{\ldots\}  & {amssymb} 或 {eufrak}  \\
  $\mathbb{ABCDE}$                 & {mathbb}\{\ldots\}    & {amssymb} 仅提供大写字母 \\
  \hline
  \end{tabular}
\end{table}
\end{document}
