% 本文档为了测试在字体要求:正文中,中文使用宋体,数字和字母使用Times New Roman体。
% 在参考文献格式要求中并没有要求公式的字体(做了一番调查,每个人对规范的解读不一样)。下面暂且认为公式也需要Times New Roman表示非文本文字,先调查了word自带公式编辑器和 mathtype:
% 1.word自带公式编辑器默认使用Cambria Math,是微软 Office 预装的数学字体。要改Times New Roman需要先在公式界面选择“ab文本”,才能在“开始->选择Times New Roman字体“修改,还有根据需要改成斜体或者加粗。也可以安装XITS Math后在公式选项修改公式区默认字体为XITS Math。两者差异较小,但相同字号下实际大小有差异。
% 2.mathtype可以自定义每一字符的字体,其使用Symbol字体作为默认的希腊字母字体,若改为Times New Roman,一些希腊字符不能输入,如\phi会显示打叉。
% 上面涉及的字体,查了两个GPT:
%%%%%%%%%%%%%%%%%%%%%%%%%%%%%%%%%%%%%%%%%%%%%%%%%%%%%%%%%GPT内容%%%%%%%%%%%%%%%%%%%%%%%%%%%%%%%%%%%%%%%%%%%%%%%%%%%%%%%%%%%%%
% 百度:
% XITS Math
% XITS Math是一款开源数学字体,由GitHub上的Alif Type项目提供。它专门设计用于数学公式排版,注重数学符号的美观与易读性。XITS Math支持丰富的数学符号和希腊字母,适用于需要高质量数学排版的应用场景‌12。

% Symbol
% Symbol字体主要用于表示各种符号,包括数学符号、单位符号等。它包含的字符范围有限,主要用于特定符号的表示,不适合用于大段文本排版。Symbol字体在数学和科学计算中常用于表示特定的数学符号和单位‌3。

% Cambria Math
% Cambria Math是Microsoft Office中的默认数学字体,专门用于排版数学公式。它注重数学公式的美观与易读性,支持广泛的字号范围和丰富的数学符号。Cambria Math是Microsoft Office的标准字体,无需额外购买,适用于Word、PPT等办公软件中的数学公式编辑‌34。

% Times New Roman
% Times New Roman是一款常用的标点字体,适用于正文排版。它具有多种字重,包括超细体、细体、常规、中黑体和粗体,适用于普通文本的阅读舒适度。Times New Roman支持基本的拉丁字母、数字和标点符号,广泛应用于文档、网页等需要清晰阅读的场景‌

% chatGPT:
% 以下是 **XITS Math**、**Symbol**、**Cambria Math** 和 **Times New Roman** 这几种字体的区别:

% 1. **XITS Math**:
%    - **类型**:数学字体。
%    - **特点**:XITS Math 是为 LaTeX 环境中的数学排版设计的一种字体,灵感来自 Times 字体系列。它特别适用于数学公式的排版,具有清晰且符合排版标准的外观。XITS Math 是自由字体,设计时考虑了数学符号的可读性和美观,适用于许多学术出版物,特别是数学、物理、工程类的文档。
%    - **用途**:主要用于数学公式和符号的排版。

% 2. **Symbol**:
%    - **类型**:符号字体。
%    - **特点**:Symbol 字体主要包含各种符号字符,包括数学符号、希腊字母等。它的设计是基于传统的符号字体,专门用于表示特殊符号而非正文文本。其字符形状较为简单,具有古典的风格。
%    - **用途**:用于数学符号和一些特殊符号的显示。

% 3. **Cambria Math**:
%    - **类型**:数学字体。
%    - **特点**:Cambria Math 是 Microsoft 为其 Office 软件套件(如 Word)设计的专门数学字体。它有较好的兼容性,尤其是在 Office 文档中处理数学公式时,具有非常高的可读性。与 Times New Roman 不同,Cambria Math 在设计时注重数学公式的清晰排版,尤其是带有上标、下标、分数等结构时,显得更为清晰和专业。
%    - **用途**:适合用于 Word 或类似文档编辑软件中的数学公式,通常在 Microsoft 环境中广泛使用。

% 4. **Times New Roman**:
%    - **类型**:衬线字体(Serif)。
%    - **特点**:Times New Roman 是一种常见的衬线字体,设计上比较传统,适合一般正文排版。它广泛用于文章、报告和书籍的正文排版中。虽然它在处理公式时也可以使用,但相比其他专门的数学字体(如 XITS Math 或 Cambria Math),它的数学符号显示并不特别优化。
%    - **用途**:主要用于普通正文的排版,也可以用于简单的数学公式,但不如专用数学字体那样适合复杂公式。

% ### 总结:
% - **XITS Math** 和 **Cambria Math** 是为数学公式设计的字体,具有优异的数学符号显示效果,尤其是在处理复杂公式时。
% - **Symbol** 主要用于符号字符,如希腊字母和数学符号,但不适用于整个数学公式的排版。
% - **Times New Roman** 是一种经典的正文字体,适合常规文本排版,但对于数学公式的排版不如专门的数学字体。

% 如果你需要专门排版数学公式,建议使用 **XITS Math** 或 **Cambria Math**。如果只是进行普通文本排版,**Times New Roman** 可能会更合适。
%%%%%%%%%%%%%%%%%%%%%%%%%%%%%%%%%%%%%%%%%%%%%%%%%%%%%%%%%GPT内容%%%%%%%%%%%%%%%%%%%%%%%%%%%%%%%%%%%%%%%%%%%%%%%%%%%%%%%%%%%%%
% 再使用GPT搜索:在数学字体中使用Times New Roman的方法有哪些,可以列出来并给出对比吗?经过一番对比后得到如下结论:
% 我们要么使用latex默认的那套、由高德纳设计制作的 Computer Modern 字体,要么使用使用unicode-math宏包设置XITS Math字体。
\documentclass{article}
\usepackage{xcolor} % 为了改颜色显示
\usepackage{enumitem} %为了枚举
\usepackage[no-math]{fontspec}        % 用于设置正文字体。使用命令\setmainfont
\usepackage{xeCJK}           % 用于处理中文字体。使用命令\setCJKmainfont
% 测试过以下都不合适
% \usepackage{times}  
% \usepackage{mathptmx}        % 设置数学公式字体为 Times 风格,不好加粗
% \usepackage{newtxmath} % incompatible with amssymb  
% AMS 宏集合是美国数学学会 (American Mathematical Society) 提供的对 LATEX 原生的数学公式排版的扩展,其核心是 amsmath 宏  包,对多行公式的排版提供了有力的支持。此外,amsfonts 宏包以及基于它的 amssymb 宏包提  供了丰富的数学符号;amsthm 宏包扩展了 LATEX 定理证明格式。
% 以下包为了打印符号表的内容引入。
\usepackage{amsmath}  % \boldsymbol
\usepackage{mathrsfs}  % 
\usepackage{latexsym}  
\usepackage{yhmath}   % \wideparen
% \usepackage{eucal}   % \EuScript
\usepackage{booktabs}  % \begin{table}
\usepackage{amssymb} % \mathbb   {{AmS} 符号}一节中大量使用

% 设置正文字体
\setmainfont{Times New Roman} % 正文数字和字母使用 Times New Roman
% \setmainfont[Mapping=tex-text]{Times New Roman}
% 设置中文正文字体为 SimSun
\setCJKmainfont{SimSun}       % 正文中文使用 SimSun

%Unicode 数学字体是一类 OpenType 字体,包含了 Unicode 字符集中的数学符号部分,字  体中也设定了数学公式排版所需的一些参数。在 xelatex 或者 lualatex 编译命令下,借助  unicode-math 宏包可以调用 Unicode 数学字体配置数学公式的字体风格。

%lshort书中称“ unicode-math 宏包与传统数学字体、符号包并不兼容,但其本身已经提供了大量的符号和字体样式。”,经过实际摸索发现只要不使用bm包,兼容性问题可以忽略(目前只发现少部分命令不支持,类似https://github.com/latex3/unicode-math/issues/411)。简而言之大部分符号都有,若需要更多符号,可以再仔细研究差异,考虑改为默认Computer Modern字体(不再包含unicode-math包和不再使用\setmathfont命令)。否则不需要关心。
\usepackage{unicode-math} % 会自动包含amsmath, 能使用大部分符号,若不包含amssymb,也能使用amssymb中一部分符号。使用\setmathfont 命令。若不使用该包,则默认Computer Modern 字体,这就是2024年以前的模板的字体。注意:unicode-math和加粗包bm冲突,使用\boldsymbol加粗。或者:The five new symbol font commands that behave in this way are: \symup, \symit, \symbf, \symsf, and \symtt. These commands switch to single-letter mathematical symbols (generally within the same OpenType font).
% unicode-math包文档说明:https://ctan.org/pkg/unicode-math


% 设置数学公式字体为 XITS Math。即Times风格。其实默认的公式字体
\setmathfont{XITS Math} % If you do not load an OpenType maths font before \begin{document}, Latin Modern Math will be loaded automatically. 
% XITS Math字体文件:https://github.com/aliftype/xits 或 https://fontlibrary.org/en/font/xits-math 双击安装即可。
\setmathfont{XITS Math}[range={cal, bfcal}, StylisticSet=1]

%%%%%%%%%%%%下面的定义是lshort书中符号表所需要的定义%%%%%%%%%%%%%%%%%%%%%%%%%%%%%%%
% Definitions
\def\lsym{$\mathsurround=0pt {}^\ell$}
\def\LSYM    #1{$#1$     & \texttt{\string#1}\lsym}

\def\SYM     #1{$#1$     & \texttt{\string#1}}
\def\BIGSYM  #1{$#1$     & $\displaystyle #1$ & \texttt{\string#1}}
\def\ACC   #1#2{$#1{#2}$ & \texttt{\string#1}*{#2}}
\def\DEL     #1{$\big#1 \bigg#1$ & \texttt{\string#1}}

\def\AMSSYM  #1{$#1$     & \texttt{\string#1}}
% These symbols rely on `amsmath' package
\def\AMSM    #1{$#1$     & \textcolor{blue}{\texttt{\string#1}}}
\def\AMSACC#1#2{$#1{#2}$ & \textcolor{blue}{\texttt{\string#1}}*{#2}}
\def\AMSBIG  #1{$#1$     & $\displaystyle #1$ & \textcolor{blue}{\texttt{\string#1}}}

\def\SC      #1{#1       & \texttt{\string#1}}

\newenvironment{symbols}[1]%
  {\small\def\arraystretch{1.5
  }
  \begin{tabular}{@{}#1@{}}}%
  {\end{tabular}}
%%%%%%%%%%%%%%%%%%%%%%%%%%%%%%%%%%%%%%%%%%%%%%%%%%%%%%%%%%%%%%%%%%%%%%%%%%%%%%%%%%%%%%%%

% 测试文档开始
\begin{document}
%%%%%%%%%%%%%%%%%%%%%%%%%%%%%%%
这是正文部分,中文使用 SimSun 字体,英文使用 Times New Roman 字体。


这是一个数学公式:
\begin{equation}
a = b + c
\end{equation}

对于XITS Math字体是否可以满足“使用Times New Roman字体”的要求,搜索资料发现区别仅仅是下面表格中的两个符号:

\begin{tabular}{@{}lcc@{}}
  \toprule 
  & \multicolumn{2}{c@{}}{``Times''-like font}\\
  \cmidrule(l){2-3}
                    & Times & Times New Roman \\
  \midrule
  Text-italic ``z'' & \setmainfont{Times}[ItalicFont={Times Italic}] \textit{z}
                    & \setmainfont{Times New Roman} \textit{z}\\   
  Percent symbol    & \setmainfont{Times} \% 
                    & \setmainfont{Times New Roman} \% \\
  \bottomrule
\end{tabular}

再次写入英文字母进行测试,用于和word文件 testword.docx 对比。

文本小写字母:

abcdefghijklmnopqrstuvwxyz

\textbf{abcdefghijklmnopqrstuvwxyz}

\textit{abcdefghijklmnopqrstuvwxyz}

\textbf{\textit{abcdefghijklmnopqrstuvwxyz}}

1.	word中默认公式编辑器(分默认Cambria Math和XITS Math):
按照斜体、斜体加粗、直体、直体加粗顺序

英文字母小写:
\begin{equation}
  abcdefghijklmnopqrstuvwxyz 
\end{equation}
\begin{equation}
  \boldsymbol{abcdefghijklmnopqrstuvwxyz} 
\end{equation}
\begin{equation}
  \mathrm{abcdefghijklmnopqrstuvwxyz}
\end{equation}
\begin{equation}
  \mathbf{abcdefghijklmnopqrstuvwxyz}
\end{equation}

英文字母大写:
\begin{equation}
  ABCDEFGHIJKLMNOPQRSTUVWXYZ 
\end{equation}
\begin{equation}
  \boldsymbol{ABCDEFGHIJKLMNOPQRSTUVWXYZ} 
\end{equation}
\begin{equation}
  \mathrm{ABCDEFGHIJKLMNOPQRSTUVWXYZ}
\end{equation}
\begin{equation}
  \mathbf{ABCDEFGHIJKLMNOPQRSTUVWXYZ}
\end{equation}
\clearpage
希腊小写
% \omicron or \o
\begin{equation}
    \alpha, \beta, \gamma, \delta, \varepsilon, \epsilon, \zeta, \eta, \theta, \vartheta, \iota, \kappa, \lambda, \mu, \nu, \xi, \omicron, \pi ,\varpi, \rho, \varrho, \sigma, \varsigma, \tau, \upsilon, \varphi, \phi, \chi, \psi, \omega
\end{equation}
\begin{equation}
  \symbf{\alpha, \beta, \gamma, \delta, \varepsilon, \epsilon, \zeta, \eta, \theta, \vartheta, \iota, \kappa, \lambda, \mu, \nu, \xi, \omicron, \pi ,\varpi, \rho, \varrho, \sigma, \varsigma, \tau, \upsilon, \varphi, \phi, \chi, \psi, \omega}
\end{equation}
% 这是一个公式:$f$是斜体,$\mathrm{f}$ ,$\symup{f}$ 是直体。$\alpha$ 是斜体,$\upalpha$ ,$\symup{\alpha}$ 是直体。
% 注意:字体用\symup{} 而不是\mathup 或者 \mathrm
\begin{equation}
  \symup{\alpha, \beta, \gamma, \delta, \varepsilon, \epsilon, \zeta, \eta, \theta, \vartheta, \iota, \kappa, \lambda, \mu, \nu, \xi, \omicron, \pi ,\varpi, \rho, \varrho, \sigma, \varsigma, \tau, \upsilon, \varphi, \phi, \chi, \psi, \omega}
\end{equation}
\begin{equation}
  \boldsymbol{\symup{\alpha, \beta, \gamma, \delta, \varepsilon, \epsilon, \zeta, \eta, \theta, \vartheta, \iota, \kappa, \lambda, \mu, \nu, \xi, \omicron, \pi ,\varpi, \rho, \varrho, \sigma, \varsigma, \tau, \upsilon, \varphi, \phi, \chi, \psi, \omega}} % symbf不起作用
\end{equation}

希腊大写
\begin{equation}
  \Alpha, \Beta, \Gamma, \Delta, \Epsilon, \Zeta, \Eta, \Theta, \Iota, \Kappa, \Lambda, \Mu, \Nu, \Xi, \Omicron, \Pi, \Rho, \Sigma, \Tau, \Upsilon, \Phi, \Chi, \Psi, \Omega
\end{equation}
\begin{equation}
  \symbf{\Alpha, \Beta, \Gamma, \Delta, \Epsilon, \Zeta, \Eta, \Theta, \Iota, \Kappa, \Lambda, \Mu, \Nu, \Xi, \Omicron, \Pi, \Rho, \Sigma, \Tau, \Upsilon, \Phi, \Chi, \Psi, \Omega}
\end{equation}
\begin{equation}
  \symit{\Alpha, \Beta, \Gamma, \Delta, \Epsilon, \Zeta, \Eta, \Theta, \Iota, \Kappa, \Lambda, \Mu, \Nu, \Xi, \Omicron, \Pi, \Rho, \Sigma, \Tau, \Upsilon, \Phi, \Chi, \Psi, \Omega}
\end{equation}
\begin{equation}
  \boldsymbol{\symit{\Alpha, \Beta, \Gamma, \Delta, \Epsilon, \Zeta, \Eta, \Theta, \Iota, \Kappa, \Lambda, \Mu, \Nu, \Xi, \Omicron, \Pi, \Rho, \Sigma, \Tau, \Upsilon, \Phi, \Chi, \Psi, \Omega}} % symbf不起作用
\end{equation}

2.word中mathtype公式编辑器见word文件test。

数学字母字体:

在LaTeX中,下面这些命令用于改变数学公式的字体样式:

‌mathtt‌:将数学符号或文本设置为等宽字体(typewriter font)。这种字体通常用于代码或需要等宽字符间距的场合。例如,$\mathtt{ABC123}$会产生等宽字体的ABC123。

‌mathsf‌:将数学符号或文本设置为无衬线字体(sans-serif font)。这种字体通常看起来更现代、简洁。例如,$\mathsf{ABC123}$会产生无衬线字体的ABC123。

‌mathbf‌:将数学符号或文本设置为加粗字体(bold font)。这种字体通常用于强调或突出显示。例如,$\mathbf{ABC123}$会产生加粗字体的ABC123。

‌mathit‌:将数学符号或文本设置为斜体(italic font)。在数学排版中,斜体常用于表示变量。例如,$\mathit{ABC123}$会产生斜体字体的ABC123。

‌mathrm‌:将数学符号或文本设置为罗马字体(roman font)。这种字体是数学排版中的默认字体,用于表示函数名、常量等。例如,$\mathrm{ABC123}$会产生罗马字体的ABC123。

‌mathnormal‌:这个命令在LaTeX中实际上并不直接改变字体样式,而是用于恢复数学模式中的默认字体(通常是斜体)。然而,在某些情况下,它可能被用来确保数学符号或文本以正常(即非加粗、非等宽等)的样式显示。例如,$\mathnormal{ABC123}$通常会产生斜体(默认样式)的ABC123。

‌mathbb、mathfrak、mathscr、mathcal都是LaTeX中用于数学公式的特殊字体命令‌。

‌mathbb‌:表示黑板粗体(blackboard bold),常用于表示数学中的集合符号,如实数集R、自然数集N等。例如,$\mathbb{R}$会产生一个黑板粗体的R‌

‌mathfrak‌:表示德文尖角体或哥特体(fraktur),这种字体在数学中不常用,但在某些特定领域或文献中可能会见到。例如,$\mathfrak{A}$会产生一个德文尖角体的A‌

‌mathscr‌:表示花体(script),这种字体常用于表示数学中的某些特定符号或集合,如变换、空间等。例如,$\mathscr{X}$会产生一个花体的X‌。需要注意的是,mathscr字体可能需要额外的宏包支持,如mathrsfs‌


‌mathcal‌:也表示一种花体(calligraphy),但与mathscr不同。mathcal字体常用于表示数学中的某些特定符号或函数,如损失函数L、时间复杂度O等。例如,$\mathcal{L}$会产生一个花体的L‌


下表为lshort中文版中,表 4.2: 数学字母字体的内容:

不使用unicode-math包和setmathfont{XITS Math} 时,包含eucal和mathrsfs包可以实现和原文一模一样的效果。
\begin{table}[htp]
  \centering
  \caption{数学字母字体} \label{tbl:math-fonts}
  \begin{tabular}{*{3}{l}}
  \hline
  \textbf{示例}    & \textbf{命令} & \textbf{依赖的宏包}\\
  \hline
  $\mathnormal{ABCDE abcde 1234}$  & {mathnormal}\{\ldots\}&       \\
  $\mathrm{ABCDE abcde 1234}$      & {mathrm}\{\ldots\}    &       \\
  $\mathit{ABCDE abcde 1234}$      & {mathit}\{\ldots\}    &       \\
  $\mathbf{ABCDE abcde 1234}$      & {mathbf}\{\ldots\}    &       \\
  $\mathsf{ABCDE abcde 1234}$      & {mathsf}\{\ldots\}    &       \\
  $\mathtt{ABCDE abcde 1234}$      & {mathtt}\{\ldots\}    &       \\
  $\mathcal{ABCDE}$                  & {mathcal}\{\ldots\}   &     \\ 
  % $\CMcal{ABCDE}$                  & {mathcal}\{\ldots\}   & 仅提供大写字母,由eucal提供 \\
  \hline
  % $\EuScript{ABCDE}$               & {mathcal}\{\ldots\}   & {eucal} 仅提供大写字母 \\
  $\mathscr{ABCDE}$                & {mathscr}\{\ldots\}   & {mathrsfs} 仅提供大写字母\\
  $\mathfrak{ABCDE abcde 1234}$    & {mathfrak}\{\ldots\}  & {amssymb} 或 {eufrak}  \\
  $\mathbb{ABCDE}$                 & {mathbb}\{\ldots\}    & {amssymb} 仅提供大写字母 \\
  \hline
  \end{tabular}
\end{table}

但使用unicode-math包和setmathfont{XITS Math} 后,上表略有不同。不过新的设置下可以取消 eucal 和 mathrsfs 包,命令改为:
\begin{table}[htp]
  \centering
  \caption{数学字母字体} \label{tbl:math-fonts}
  \begin{tabular}{*{3}{l}}
  \hline
  \textbf{示例}    & \textbf{命令} & \textbf{依赖的宏包}\\
  \hline
  $\symnormal{ABCDE abcde 1234}$  & {symnormal}*{\ldots}&       \\
  $\symrm{ABCDE abcde 1234}$      & {symrm}*{\ldots}    &       \\
  $\symit{ABCDE abcde 1234}$      & {symit}*{\ldots}    &       \\
  $\symbf{ABCDE abcde 1234}$      & {symbf}*{\ldots}    &       \\
  $\symsf{ABCDE abcde 1234}$      & {symsf}*{\ldots}    &       \\
  $\symtt{ABCDE abcde 1234}$      & {symtt}*{\ldots}    &       \\
  $\symcal{ABCDE}$                  & {symcal}*{\ldots}   &   \\
  \hline
  % $\EuScript{ABCDE}$               & {symcal}*{\ldots}   & {eucal} 仅提供大写字母 \\
  $\symscr{ABCDE}$                & {symscr}*{\ldots}   & \\
  $\symfrak{ABCDE abcde 1234}$    & {symfrak}*{\ldots}  &   \\
  $\symbb{ABCDE}$                 & {symbb}*{\ldots}    &    \\
  \hline
  \end{tabular}
\end{table}
注:效果有差异,但够用。


其中常用:

黑板粗体(blackboard bold)命令mathbb
\begin{equation}
  \mathbb{ABCDEFGHIJKLMNOPQRSTUVWXYZ}
\end{equation}
unicode-math包的黑板粗体命令symbb
\begin{equation}
  \symbb{ABCDEFGHIJKLMNOPQRSTUVWXYZ}
\end{equation}
花体(calligraphy) 命令mathcal
\begin{equation}
  \mathcal{ABCDEFGHIJKLMNOPQRSTUVWXYZ}
\end{equation}
unicode-math包的花体命令symcal
\begin{equation}
  \symcal{ABCDEFGHIJKLMNOPQRSTUVWXYZ}
\end{equation}


另一种花体(script) 命令mathscr
\begin{equation}
  \mathscr{ABCDEFGHIJKLMNOPQRSTUVWXYZ}
\end{equation}
unicode-math包的花体命令symscr
\begin{equation}
  \symscr{ABCDEFGHIJKLMNOPQRSTUVWXYZ}
\end{equation}
  

\clearpage
\section{符号表}\label{sec:math-tables}
  \begin{enumerate}
    \item \textcolor{blue}{蓝色}的命令依赖 {amsmath} 宏包(非 {amssymb} 宏包);
    \item 带有角标\lsym 的符号命令依赖 {latexsym} 宏包。
  \end{enumerate}
注:对于unicode-math包设置XITS Math,只是unicode-math取消了{arrowvert},{Arrowvert},{bracevert} 的支持,其他符号和默认的Computer Modern字体时一样都支持。
\subsection{\LaTeX{} 普通符号}
\begin{table}[htp]
\centering
\caption{文本/数学模式通用符号}\label{tbl:general-syms}
\begin{quote}\footnotesize%
这些符号可用于文本和数学模式。
\end{quote}
\begin{symbols}{*4{cl}}
\hline
 \SC{\{}    &  \SC{\}}  &  \SC{\$}         &  \SC{\%}               \\
 \SC{\dag}  &  \SC{\S}  &  \SC{\copyright} &  \SC{\dots}            \\
 \SC{\ddag} &  \SC{\P}  &  \SC{\pounds}    &                        \\
\hline
\end{symbols}
\end{table}

\begin{table}[htp]
\centering
\caption{希腊字母} \label{tbl:math-greek}
\begin{quote}\footnotesize%
{Alpha},{Beta} 等希腊字母符号不存在,因为它们和拉丁字母 A,B 等一模一样;
小写字母里也不存在 {omicron},直接用拉丁字母 $o$ 代替。
\end{quote}
\begin{symbols}{*4{cl}}
\hline
 \SYM{\alpha}     & \SYM{\theta}     & \SYM{o}          & \SYM{\upsilon}  \\
 \SYM{\beta}      & \SYM{\vartheta}  & \SYM{\pi}        & \SYM{\phi}      \\
 \SYM{\gamma}     & \SYM{\iota}      & \SYM{\varpi}     & \SYM{\varphi}   \\
 \SYM{\delta}     & \SYM{\kappa}     & \SYM{\rho}       & \SYM{\chi}      \\
 \SYM{\epsilon}   & \SYM{\lambda}    & \SYM{\varrho}    & \SYM{\psi}      \\
 \SYM{\varepsilon}& \SYM{\mu}        & \SYM{\sigma}     & \SYM{\omega}    \\
 \SYM{\zeta}      & \SYM{\nu}        & \SYM{\varsigma}  &                 \\
 \SYM{\eta}       & \SYM{\xi}        & \SYM{\tau}       &                 \\[1ex]
 \SYM{\Gamma}     & \SYM{\Lambda}    & \SYM{\Sigma}     & \SYM{\Psi}      \\
 \SYM{\Delta}     & \SYM{\Xi}        & \SYM{\Upsilon}   & \SYM{\Omega}    \\
 \SYM{\Theta}     & \SYM{\Pi}        & \SYM{\Phi}       &                 \\[1ex]
 \AMSM{\varGamma} & \AMSM{\varLambda}& \AMSM{\varSigma}  & \AMSM{\varPsi}      \\
 \AMSM{\varDelta} & \AMSM{\varXi}    & \AMSM{\varUpsilon}& \AMSM{\varOmega}    \\
 \AMSM{\varTheta} & \AMSM{\varPi}    & \AMSM{\varPhi}    &                 \\
\hline
\end{symbols}
\end{table}

\begin{table}[htp]
\centering
\caption{二元关系符} \label{tbl:math-rel}
\begin{quote}\footnotesize%
所有的二元关系符都可以加 {not} 前缀得到相反意义的关系符,例如 {not}\texttt{=} 就得到不等号(同 {ne})。
\end{quote}
\begin{symbols}{*3{cl}}
\hline
 \SYM{<}              & \SYM{>}                    & \SYM{=}          \\
 \SYM{\leq} or {le}   & \SYM{\geq} or {ge} & \SYM{\equiv}     \\
 \SYM{\ll}            & \SYM{\gg}                  & \SYM{\doteq}     \\
 \SYM{\prec}          & \SYM{\succ}                & \SYM{\sim}       \\
 \SYM{\preceq}        & \SYM{\succeq}              & \SYM{\simeq}     \\
 \SYM{\subset}        & \SYM{\supset}              & \SYM{\approx}    \\
 \SYM{\subseteq}      & \SYM{\supseteq}            & \SYM{\cong}      \\
 \LSYM{\sqsubset}     & \LSYM{\sqsupset}           & \LSYM{\Join}     \\
 \SYM{\sqsubseteq}    & \SYM{\sqsupseteq}          & \SYM{\bowtie}    \\
 \SYM{\in}            & \SYM{\ni}, {owns}      & \SYM{\propto}    \\
 \SYM{\vdash}         & \SYM{\dashv}               & \SYM{\models}    \\
 \SYM{\mid}           & \SYM{\parallel}            & \SYM{\perp}      \\
 \SYM{\smile}         & \SYM{\frown}               & \SYM{\asymp}     \\
 \SYM{:}              & \SYM{\notin}               & \SYM{\neq} or {ne} \\
\hline
\end{symbols}
\end{table}

\begin{table}[htp]
\centering
\caption{二元运算符}\label{tbl:math-op}
\begin{symbols}{*3{cl}}
\hline
 \SYM{+}              & \SYM{-}              &                     \\
 \SYM{\pm}            & \SYM{\mp}            & \SYM{\triangleleft} \\
 \SYM{\cdot}          & \SYM{\div}           & \SYM{\triangleright}\\
 \SYM{\times}         & \SYM{\setminus}      & \SYM{\star}         \\
 \SYM{\cup}           & \SYM{\cap}           & \SYM{\ast}          \\
 \SYM{\sqcup}         & \SYM{\sqcap}         & \SYM{\circ}         \\
 \SYM{\vee}, {lor}& \SYM{\wedge},{land}  & \SYM{\bullet}   \\
 \SYM{\oplus}         & \SYM{\ominus}        & \SYM{\diamond}      \\
 \SYM{\odot}          & \SYM{\oslash}        & \SYM{\uplus}        \\
 \SYM{\otimes}        & \SYM{\bigcirc}       & \SYM{\amalg}        \\
 \SYM{\bigtriangleup} &\SYM{\bigtriangledown}& \SYM{\dagger}       \\
 \LSYM{\lhd}          & \LSYM{\rhd}          & \SYM{\ddagger}      \\
 \LSYM{\unlhd}        & \LSYM{\unrhd}        & \SYM{\wr}           \\
\hline
\end{symbols}
\end{table}

\begin{table}[htp]
\centering
\caption{巨算符}\label{tbl:math-bigop}
\def\arraystretch{2.2}
\begin{symbols}{*3{ccl}}
\hline
 \BIGSYM{\sum}      & \BIGSYM{\bigcup}   & \BIGSYM{\bigvee}  \\
 \BIGSYM{\prod}     & \BIGSYM{\bigcap}   & \BIGSYM{\bigwedge} \\
 \BIGSYM{\coprod}   & \BIGSYM{\bigsqcup} & \BIGSYM{\biguplus} \\
 \BIGSYM{\int}      & \BIGSYM{\oint}     & \BIGSYM{\bigodot} \\
 \BIGSYM{\bigoplus} & \BIGSYM{\bigotimes} & \\
 \AMSBIG{\iint}     & \AMSBIG{\iiint}    & \AMSBIG{\iiiint}  \\
 \AMSBIG{\idotsint} &                    & \\
\hline
\end{symbols}
\end{table}

\begin{table}[htp]
\centering
\caption{数学重音符号}\label{tbl:math-accents}
\begin{quote}\footnotesize%
最后一个 {wideparen} 依赖 {yhmath} 宏包。
\end{quote}
\begin{symbols}{*3{cl}}
\hline
\ACC{\hat}{a}   & \ACC{\check}{a} & \ACC{\tilde}{a}       \\
\ACC{\acute}{a} & \ACC{\grave}{a} & \ACC{\breve}{a}       \\
\ACC{\bar}{a}   & \ACC{\vec}{a}   & \ACC{\mathring}{a}    \\
\ACC{\dot}{a}   & \ACC{\ddot}{a}  & \AMSACC{\dddot}{a}    \\
\AMSACC{\ddddot}{a} \\[1ex]
\ACC{\widehat}{AAA} & \ACC{\widetilde}{AAA} & \ACC{\wideparen}{AAA} \\
\hline
\end{symbols}
\end{table}

\begin{table}[htp]
\centering
\caption{箭头} \label{tbl:math-arrows}
\begin{symbols}{*2{cl}}
\hline
 \SYM{\leftarrow} or {gets} & \SYM{\longleftarrow} \\
 \SYM{\rightarrow} or {to}   & \SYM{\longrightarrow} \\
 \SYM{\leftrightarrow}    & \SYM{\longleftrightarrow} \\
 \SYM{\Leftarrow}         & \SYM{\Longleftarrow}     \\
 \SYM{\Rightarrow}        & \SYM{\Longrightarrow}    \\
 \SYM{\Leftrightarrow}    & \SYM{\Longleftrightarrow}\\
 \SYM{\mapsto}            & \SYM{\longmapsto}        \\
 \SYM{\hookleftarrow}     & \SYM{\hookrightarrow}    \\
 \SYM{\leftharpoonup}     & \SYM{\rightharpoonup}    \\
 \SYM{\leftharpoondown}   & \SYM{\rightharpoondown}  \\
 \SYM{\rightleftharpoons} & \SYM{\iff}               \\
 \SYM{\uparrow}           & \SYM{\downarrow} \\
 \SYM{\updownarrow}       & \SYM{\Uparrow} \\
 \SYM{\Downarrow}         & \SYM{\Updownarrow} \\
 \SYM{\nearrow}           & \SYM{\searrow} \\
 \SYM{\swarrow}           & \SYM{\nwarrow} \\
 \LSYM{\leadsto}          &              \\
\hline
\end{symbols}
\end{table}

\begin{table}[htp]
\centering
\caption{作为重音的箭头符号}  \label{tbl:math-arrow-accents}
\begin{symbols}{*2{cl}}
\hline
\ACC{\overrightarrow}{AB}     & \AMSACC{\underrightarrow}{AB}     \\
\ACC{\overleftarrow}{AB}      & \AMSACC{\underleftarrow}{AB}      \\
\AMSACC{\overleftrightarrow}{AB} & \AMSACC{\underleftrightarrow}{AB} \\
\hline
\end{symbols}
\end{table}

\begin{table}[htp]
\centering
\caption{定界符}\label{tbl:math-delims}
\begin{quote}\footnotesize%
{amsmath} 还定义了 {lvert}、{rvert} 和 {lVert}、{rVert},
分别作为 {vert} 和 {Vert} 对应的开符号(左侧)和闭符号(右侧)的命令。
\end{quote}
\begin{symbols}{*4{cl}}
\hline
 \SYM{(}                  & \SYM{)}                  & \SYM{\uparrow}     & \SYM{\downarrow}   \\
 \SYM{[}  or {lbrack} & \SYM{]}  or {rbrack} & \SYM{\Uparrow}     & \SYM{\Downarrow}   \\
 \SYM{\{} or {lbrace} & \SYM{\}} or {rbrace} & \SYM{\updownarrow} & \SYM{\Updownarrow} \\
 \SYM{|}  or {vert}   & \SYM{\|} or {Vert}   & \SYM{\lceil}       & \SYM{\rceil}       \\
 \SYM{\langle}            & \SYM{\rangle}            & \SYM{\lfloor}      & \SYM{\rfloor}      \\
 \SYM{/}                  & \SYM{\backslash} \\
\hline
\end{symbols}
\end{table}

\begin{table}[htp]
\centering
\caption{用于行间公式的大定界符}\label{tbl:math-large-delims}
\def\arraystretch{1.8}
\begin{symbols}{*3{cl}}
\hline
 \DEL{\lgroup}      & \DEL{\rgroup}      & \DEL{\lmoustache}  \\
%  \DEL{\arrowvert}   & \DEL{\Arrowvert}   & \DEL{\bracevert} \\  
 \DEL{\rmoustache} \\
\hline
\end{symbols}
\end{table}

\begin{table}[htp]
\centering
\caption{其他符号}\label{tbl:math-misc}
\begin{symbols}{*4{cl}}
\hline
 \SYM{\dots}       & \SYM{\cdots}      & \SYM{\vdots}      & \SYM{\ddots}     \\
 \SYM{\hbar}       & \SYM{\imath}      & \SYM{\jmath}      & \SYM{\ell}       \\
 \SYM{\Re}         & \SYM{\Im}         & \SYM{\aleph}      & \SYM{\wp}        \\
 \SYM{\forall}     & \SYM{\exists}     & \LSYM{\mho}       & \SYM{\partial}   \\
 \SYM{'}           & \SYM{\prime}      & \SYM{\emptyset}   & \SYM{\infty}     \\
 \SYM{\nabla}      & \SYM{\triangle}   & \LSYM{\Box}       & \LSYM{\Diamond}  \\
 \SYM{\bot}        & \SYM{\top}        & \SYM{\angle}      & \SYM{\surd}      \\
 \SYM{\diamondsuit} & \SYM{\heartsuit} & \SYM{\clubsuit}   & \SYM{\spadesuit} \\
 \SYM{\neg} or {lnot} & \SYM{\flat} & \SYM{\natural}    & \SYM{\sharp}     \\
\hline
\end{symbols}
\end{table}

\clearpage
\subsection{{AMS} 符号}

本小节所有符号依赖 {amssymb} 宏包。

\begin{table}[htp]
\centering
\caption{\AmS{} 希腊字母和希伯来字母} \label{tbl:ams-greek-hebrew}
\begin{symbols}{*5{cl}}
\hline
\AMSSYM{\digamma}   &\AMSSYM{\varkappa} & \AMSSYM{\beth} &\AMSSYM{\gimel} & \AMSSYM{\daleth}\\
\hline
\end{symbols}
\end{table}

\begin{table}[htp]
\centering
\caption{\AmS{} 二元关系符} \label{tbl:ams-rel}
\begin{symbols}{*3{cl}}
\hline
 \AMSSYM{\lessdot}           & \AMSSYM{\gtrdot}            & \AMSSYM{\doteqdot} \\ %必须包含amssymb
 \AMSSYM{\leqslant}          & \AMSSYM{\geqslant}          & \AMSSYM{\risingdotseq}     \\
 \AMSSYM{\eqslantless}       & \AMSSYM{\eqslantgtr}        & \AMSSYM{\fallingdotseq}    \\
 \AMSSYM{\leqq}              & \AMSSYM{\geqq}              & \AMSSYM{\eqcirc}           \\
 \AMSSYM{\lll} or {llless}& \AMSSYM{\ggg}               & \AMSSYM{\circeq}  \\
 \AMSSYM{\lesssim}           & \AMSSYM{\gtrsim}            & \AMSSYM{\triangleq}        \\
 \AMSSYM{\lessapprox}        & \AMSSYM{\gtrapprox}         & \AMSSYM{\bumpeq}           \\
 \AMSSYM{\lessgtr}           & \AMSSYM{\gtrless}           & \AMSSYM{\Bumpeq}           \\
 \AMSSYM{\lesseqgtr}         & \AMSSYM{\gtreqless}         & \AMSSYM{\thicksim}         \\ %必须包含amssymb
 \AMSSYM{\lesseqqgtr}        & \AMSSYM{\gtreqqless}        & \AMSSYM{\thickapprox}      \\ %必须包含amssymb
 \AMSSYM{\preccurlyeq}       & \AMSSYM{\succcurlyeq}       & \AMSSYM{\approxeq}         \\
 \AMSSYM{\curlyeqprec}       & \AMSSYM{\curlyeqsucc}       & \AMSSYM{\backsim}          \\
 \AMSSYM{\precsim}           & \AMSSYM{\succsim}           & \AMSSYM{\backsimeq}        \\
 \AMSSYM{\precapprox}        & \AMSSYM{\succapprox}        & \AMSSYM{\vDash}            \\
 \AMSSYM{\subseteqq}         & \AMSSYM{\supseteqq}         & \AMSSYM{\Vdash}            \\
 \AMSSYM{\shortparallel}     & \AMSSYM{\Supset}            & \AMSSYM{\Vvdash}           \\%必须包含amssymb
 \AMSSYM{\blacktriangleleft} & \AMSSYM{\sqsupset}          & \AMSSYM{\backepsilon}      \\
 \AMSSYM{\vartriangleright}  & \AMSSYM{\because}           & \AMSSYM{\varpropto}        \\%必须包含amssymb
 \AMSSYM{\blacktriangleright}& \AMSSYM{\Subset}            & \AMSSYM{\between}          \\
 \AMSSYM{\trianglerighteq}   & \AMSSYM{\smallfrown}        & \AMSSYM{\pitchfork}        \\%必须包含amssymb
 \AMSSYM{\vartriangleleft}   & \AMSSYM{\shortmid}          & \AMSSYM{\smallsmile}       \\%必须包含amssymb
 \AMSSYM{\trianglelefteq}    & \AMSSYM{\therefore}         & \AMSSYM{\sqsubset}         \\
\hline
\end{symbols}
\end{table}

\begin{table}[htp]
\centering
\caption{{AmS} 二元运算符} \label{tbl:ams-op}
\begin{symbols}{*3{cl}}
\hline
 \AMSSYM{\dotplus}        & \AMSSYM{\centerdot}      &       \\%必须包含amssymb
 \AMSSYM{\ltimes}         & \AMSSYM{\rtimes}         & \AMSSYM{\divideontimes} \\
 \AMSSYM{\doublecup}      & \AMSSYM{\doublecap}      & \AMSSYM{\setminus} \\ %by unicode-math package,  \setminus instead of \smallsetminus。%必须包含amssymb
 \AMSSYM{\veebar}         & \AMSSYM{\barwedge}       & \AMSSYM{\doublebarwedge}\\
 \AMSSYM{\boxplus}        & \AMSSYM{\boxminus}       & \AMSSYM{\circleddash}   \\
 \AMSSYM{\boxtimes}       & \AMSSYM{\boxdot}         & \AMSSYM{\circledcirc}   \\
 \AMSSYM{\intercal}       & \AMSSYM{\circledast}     & \AMSSYM{\rightthreetimes} \\
 \AMSSYM{\curlyvee}       & \AMSSYM{\curlywedge}     & \AMSSYM{\leftthreetimes} \\
\hline
\end{symbols}
\end{table}

\begin{table}[htp]
\centering
\caption{{AmS} 箭头}\label{tbl:ams-arrows}
\begin{symbols}{*2{cl}}
\hline
 \AMSSYM{\dashleftarrow}      & \AMSSYM{\dashrightarrow}     \\%必须包含amssymb
 \AMSSYM{\leftleftarrows}     & \AMSSYM{\rightrightarrows}   \\
 \AMSSYM{\leftrightarrows}    & \AMSSYM{\rightleftarrows}    \\
 \AMSSYM{\Lleftarrow}         & \AMSSYM{\Rrightarrow}        \\
 \AMSSYM{\twoheadleftarrow}   & \AMSSYM{\twoheadrightarrow}  \\
 \AMSSYM{\leftarrowtail}      & \AMSSYM{\rightarrowtail}     \\
 \AMSSYM{\leftrightharpoons}  & \AMSSYM{\rightleftharpoons}  \\
 \AMSSYM{\Lsh}                & \AMSSYM{\Rsh}                \\
 \AMSSYM{\looparrowleft}      & \AMSSYM{\looparrowright}     \\
 \AMSSYM{\curvearrowleft}     & \AMSSYM{\curvearrowright}    \\
 \AMSSYM{\circlearrowleft}    & \AMSSYM{\circlearrowright}   \\%必须包含amssymb
 \AMSSYM{\multimap}           & \AMSSYM{\upuparrows}         \\
 \AMSSYM{\downdownarrows}     & \AMSSYM{\upharpoonleft}      \\
 \AMSSYM{\upharpoonright}     & \AMSSYM{\downharpoonright}   \\
 \AMSSYM{\rightsquigarrow}    & \AMSSYM{\leftrightsquigarrow}\\
\hline
\end{symbols}
\end{table}

\begin{table}[htp]
\centering
\caption{{AmS} 反义二元关系符和箭头}\label{tbl:ams-negative}
\begin{symbols}{*3{cl}}
\hline
 \AMSSYM{\nless}           & \AMSSYM{\ngtr}            & \AMSSYM{\varsubsetneqq}    \\%必须包含amssymb
 \AMSSYM{\lneq}            & \AMSSYM{\gneq}            & \AMSSYM{\varsupsetneqq}    \\%必须包含amssymb
 \AMSSYM{\nleq}            & \AMSSYM{\ngeq}            & \AMSSYM{\nsubseteqq}       \\%必须包含amssymb
 \AMSSYM{\nleqslant}       & \AMSSYM{\ngeqslant}       & \AMSSYM{\nsupseteqq}       \\%必须包含amssymb
 \AMSSYM{\lneqq}           & \AMSSYM{\gneqq}           & \AMSSYM{\nmid}             \\
 \AMSSYM{\lvertneqq}       & \AMSSYM{\gvertneqq}       & \AMSSYM{\nparallel}        \\%必须包含amssymb
 \AMSSYM{\nleqq}           & \AMSSYM{\ngeqq}           & \AMSSYM{\nshortmid}        \\%必须包含amssymb
 \AMSSYM{\lnsim}           & \AMSSYM{\gnsim}           & \AMSSYM{\nshortparallel}   \\%必须包含amssymb
 \AMSSYM{\lnapprox}        & \AMSSYM{\gnapprox}        & \AMSSYM{\nsim}             \\
 \AMSSYM{\nprec}           & \AMSSYM{\nsucc}           & \AMSSYM{\ncong}            \\
 \AMSSYM{\npreceq}         & \AMSSYM{\nsucceq}         & \AMSSYM{\nvdash}           \\%必须包含amssymb
 \AMSSYM{\precneqq}        & \AMSSYM{\succneqq}        & \AMSSYM{\nvDash}           \\
 \AMSSYM{\precnsim}        & \AMSSYM{\succnsim}        & \AMSSYM{\nVdash}           \\
 \AMSSYM{\precnapprox}     & \AMSSYM{\succnapprox}     & \AMSSYM{\nVDash}           \\
 \AMSSYM{\subsetneq}       & \AMSSYM{\supsetneq}       & \AMSSYM{\ntriangleleft}    \\%必须包含amssymb
 \AMSSYM{\varsubsetneq}    & \AMSSYM{\varsupsetneq}    & \AMSSYM{\ntriangleright}   \\%必须包含amssymb
 \AMSSYM{\nsubseteq}       & \AMSSYM{\nsupseteq}       & \AMSSYM{\ntrianglelefteq}  \\
 \AMSSYM{\subsetneqq}      & \AMSSYM{\supsetneqq}      & \AMSSYM{\ntrianglerighteq} \\
 \AMSSYM{\nleftarrow}      & \AMSSYM{\nrightarrow}     & \AMSSYM{\nleftrightarrow}  \\
 \AMSSYM{\nLeftarrow}      & \AMSSYM{\nRightarrow}     & \AMSSYM{\nLeftrightarrow}  \\
\hline
\end{symbols}
\end{table}

\begin{table}[htp]
\centering
\caption{{AmS} 定界符}\label{tbl:ams-delims}
\begin{symbols}{*4{cl}}
\hline
\AMSSYM{\ulcorner} & \AMSSYM{\urcorner} & \AMSSYM{\llcorner} & \AMSSYM{\lrcorner} \\
\hline
\end{symbols}
\end{table}

\begin{table}[htp]
\centering
\caption{{AmS} 其它符号}\label{tbl:ams-misc}
\begin{symbols}{*3{cl}}
\hline
 \AMSSYM{\hbar}             & \AMSSYM{\hslash}           & \AMSSYM{\Bbbk}            \\
 \AMSSYM{\square}           & \AMSSYM{\blacksquare}      & \AMSSYM{\circledS}        \\%必须包含amssymb
 \AMSSYM{\vartriangle}      & \AMSSYM{\blacktriangle}    & \AMSSYM{\complement}      \\
 \AMSSYM{\triangledown}     & \AMSSYM{\blacktriangledown}& \AMSSYM{\Game}            \\
 \AMSSYM{\lozenge}          & \AMSSYM{\blacklozenge}     & \AMSSYM{\bigstar}         \\%必须包含amssymb
 \AMSSYM{\angle}            & \AMSSYM{\measuredangle}    & \\
 \AMSSYM{\diagup}           & \AMSSYM{\diagdown}         & \AMSSYM{\backprime}       \\
 \AMSSYM{\nexists}          & \AMSSYM{\Finv}             & \AMSSYM{\varnothing}      \\
 \AMSSYM{\eth}              & \AMSSYM{\sphericalangle}   & \AMSSYM{\mho}             \\
\hline
\end{symbols}
\end{table}



\end{document}
