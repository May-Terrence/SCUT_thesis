\chapter{公式字体选项——新增功能}
2024年陆续有同学提出公式数学符号字体不对,于是有了\url{https://github.com/mengchaoheng/SCUT_thesis/issues/81}的讨论——数学公式字体应该使用什么?这里简单提一下:

1)如果十分确定自己使用什么字体,或者对unicode-math的兼容性问题无法忍受,又或者持保守态度,不想冒险使用“新字体”,那不要改任何设置,使用默认的 Computer Modern 即可,因为模板的有效性已经由很多届同学验证,我反复提到这句话是因为,我们只是写毕业论文,不是成为latex专家,不要在修改模板上花时间,尽量使用现有的东西。不想折腾字体的同学可以不用继续往下看本章了。

2)若还是非常执著于修改公式字体,那可以试试所有OpenType math字体,XITS Math是我看比较合适的一种。在导言区或者cls文件加入
\begin{lstlisting}
\usepackage{unicode-math} 
\setmathfont{XITS Math}[math-style=ISO,bold-style=ISO] 
\setmathfont{XITS Math}[range={cal, bfcal}, StylisticSet=1] 
\end{lstlisting}
即实现了通过unicode-math宏包设置XITS Math字体并设置了选项[math-style=ISO,bold-style=ISO]。具体功能可以查看宏包文件。

本章摘选一些公式作为公式字体测试,同学们可以换成自己的文本,测试不同公式字体设置的效果。更多字体测试文件在math\_font文件夹。或者在\url{https://github.com/mengchaoheng/OpenType-MATH-TTF}对比多种字体效果。注意基本的符号规范,可在math\_font文件夹翻一翻基本内容。

\section{建模}
\label{sec:1}
\subsection{符号约定}
机体系的原点位于重心,正交的机体轴线表示为$\text{ }\!\!\{\!\!\text{ }{\boldsymbol{X}^{B}},{\boldsymbol{Y}^{B}},{\boldsymbol{Z}^{B}}\text{ }\!\!\}\!\!\text{ }$。 无人机沿机身轴线的角速度表示为${{\boldsymbol{\omega }}^{B}}=[p \quad q \quad r]^{T}$。风扇的转速记为$\Omega$。 每个控制舵的偏转角表示为 ${{\delta }_{i}}$,以矢量形式描述为$[{{\delta }_{1}} \quad {{\delta }_{2}} \quad {{\delta }_{3}} \quad {{\delta }_{4}}]^T$。



惯性系中的速度、惯性系中的姿态和风速分别表示为${{\boldsymbol{V}}^{I}}$,$\boldsymbol{\eta }$ 和 ${{\boldsymbol{W}}^{I}}$。 
由于涵道风扇无人机对所有机身轴对称,无人机的惯性矩阵可以简化为对角矩阵,表示为 $ \boldsymbol{I}=\text{diag}({{I}_{x} },{{I}_{y}},{{I}_{z}}) $。

在机体坐标系中表示的合力矩表示为 $\boldsymbol M$ ,为以下几项的和:
\begin{equation}
	{\boldsymbol{M}} = {{\boldsymbol{M}}_\text{vane}} + \boldsymbol{M}_{\text{flap}} + {\boldsymbol{M}}_{\text{fan}} + {\boldsymbol{M}}_{\text{aero}}
	\label{eq_1}
\end{equation}
其中 $ {{\boldsymbol{M}}_\text{vane}} $ 是从控制舵产生的控制力矩矢量。 $ \boldsymbol{M}_{\text{flap}} $ 是作用在固定气动襟翼上的反扭矩作用。 $ {\boldsymbol{M}}_{\text{fan}} $ 包括来自旋转的风扇的气动扭矩、角加速度力矩和陀螺效应。 $  {\boldsymbol{M}}_{\text{aero}} $ 是气动力矩。

\subsection{控制力矩}
四个控制舵的总控制力矩可以通过以下公式表示:
\begin{equation}
{{\boldsymbol{M}}_\text{vane}} = {k_{cv}}V_e^2\left[ {\begin{array}{*{20}{c}}
	{{\rm{ - }}{l_1}}&0&{{l_1}}&0\\
	0&{{\rm{ - }}{l_1}}&0&{{l_1}}\\
	{{l_2}}&{{l_2}}&{{l_2}}&{{l_2}}
	\end{array}} \right]{\boldsymbol{\delta }}
\label{eq_2}
\end{equation}
其中 $ {{k}_{cv}} $ 是与舵形状相关的常数系数。 $ {{l}_{1}} $ 和 $ {{l}_{2}} $ 是图中所示的杠杆臂。 ${{V}_{e}}$ 表示涵道流出的速度,由下式给出:
\begin{equation}
	{V_e} = {k_v}\Omega 
	\label{eq_3}
\end{equation}
其中 $ {{k}_{v}} $ 是与涵道的所有空气动力学特征相结合的常系数。 

如果所有舵偏转角都保持在气动失速极限内,则式 \eqref{eq_2} 中的比例关系成立,表示为 $ {{\delta }_{m}} $。 这是 ${\boldsymbol \delta}$ 的基本约束:
\begin{equation} 
	- {\delta _m} \le {\delta _i} \le {\delta _m},   i = 1,2,3,4.
	\label{eq_4}
\end{equation}

显然,式\eqref{eq_2} 将冗余舵偏转角映射到三维的控制力矩,表明无人机角速度子系统的过度驱动特性。 通过使用虚拟控制输入$ \boldsymbol{\nu }=[{\nu }_{x} \quad {\nu }_{y} \quad {\nu }_{z}]^{T}$,可以将控制力矩效应重构为模块化形式:
\begin{equation}
	\left\{ \begin{array}{l}
	{{\boldsymbol{I}}^{ - 1}}{{\boldsymbol{M}}_\text{vane}} = {{\boldsymbol{H}}_1}{\boldsymbol{\nu }}{\Omega ^2}\\
	{\boldsymbol{\nu }} = {\boldsymbol{B\delta }}
	\end{array} \right.
	\label{eq_5}
	\end{equation}
其中
	\begin{equation}
	\begin{array}{ccccc}
	{{\boldsymbol{H}}_1} \buildrel \Delta \over =   {k_{cv}}k_v^2\left[ {\begin{array}{*{20}{c}}
		{2I_x^{ - 1}{l_1}}&{}&{}\\
		{}&{2I_y^{ - 1}{l_1}}&{}\\
		{}&{}&{4I_z^{ - 1}{l_2}}
		\end{array}} \right]     \quad
	{\boldsymbol{B}} \buildrel \Delta \over =   \left[ {\begin{array}{*{20}{c}}
		{ - 0.5}&0&{0.5}&0\\
		0&{ - 0.5}&0&{0.5}\\
		{0.25}&{0.25}&{0.25}&{0.25}
		\end{array}} \right]
	\end{array}
	\label{eq_6}
\end{equation}

\section{控制器设计}


\subsection{控制分配}
对给定的 ${{\boldsymbol{\nu }}_{d}}$ 求解适当的舵偏转角 $\boldsymbol{\delta }$,
\begin{equation}
	{\boldsymbol {\nu}_d}={\boldsymbol{B\delta}}
	\label{eq_29.5}
\end{equation}
上式的解表示为 $\boldsymbol{\delta }_d$。 具体可以描述为:对于给定的${{\boldsymbol{\nu }}_{d}}$,求${{\boldsymbol{\delta }}_{d}}\in \Delta $使得 ${{\boldsymbol{\nu }}_{d}}=\boldsymbol{B}{{\boldsymbol{\delta }}_{d}}$,或者最小化分配误差 ${{\boldsymbol{\ nu }}_{d}}-\boldsymbol{B}{{\boldsymbol{\delta }}_{d}}$,其中 $\Delta $ 是允许控制集,定义为:
\begin{equation}
	\Delta=\left\{\boldsymbol{\delta} \in \mathbb{R}^{4} \mid-\delta_{m} \leq \delta_{i} \leq \delta_{m}, i=1,2,3,4\right\}
	\label{eq_30}
\end{equation}
如果 ${{\boldsymbol{\nu }}_{d}}$ 包含在可达集 (AS) 中,则称它是可达到的,可达集记为 $A$ 并定义为:
\begin{equation}
	A=\left\{\boldsymbol{\nu} \in \mathbb{R}^{3} \mid \boldsymbol{\nu}=\boldsymbol{B} \boldsymbol{\delta}, \boldsymbol{\delta} \in \Delta\right\}
	\label{eq_31}
\end{equation}

为了达到控制分配的目的,一种广泛采用的方法是直接计算 $\boldsymbol{B}$ 的伪逆:
\begin{equation}
	{{\boldsymbol{\delta }}_d} = {{\boldsymbol{B}}^\dag }{{\boldsymbol{\nu }}_d},   \quad  {{\boldsymbol{B}}^\dag } = {{\boldsymbol{B}}^T}{\left( {{\boldsymbol{B}}{{\boldsymbol{B}}^T}} \right)^{ - 1}}
	\label{eq_32}
\end{equation}

PCA 算法通过解决以下优化问题:
\begin{equation}
	\begin{aligned}
	&\max _{\alpha, \boldsymbol{\delta}} \alpha\\
	&\text { s.t. }\left\{\begin{array}{l}
	\boldsymbol{\delta} \in \Delta \\
	\boldsymbol{B} \boldsymbol{\delta}=\boldsymbol{\tau}_{i}+\alpha \boldsymbol{\tau}_{f} \\
	0 \leq \alpha \leq 1
	\end{array}\right.
	\end{aligned}
	\label{eq_pca}
\end{equation}
其中 $ \boldsymbol{B} $、$ \Delta $ 分别由式 \eqref{eq_6}、\eqref{eq_30} 定义。 $ \boldsymbol{\tau}_{i} $ 和 $ \boldsymbol{\tau}_{f} $ 在式 \eqref{eq_28} 中定义。



\section{控制理论}

\subsection{坐标变换}
考虑一个多输入多输出的非线性系统,其形式描述为
\begin{equation}
  \begin{aligned}
    \dot{\boldsymbol{x}}&=\boldsymbol{f}(\boldsymbol{x})+\boldsymbol{G}(\boldsymbol{x})\boldsymbol{u} + \boldsymbol{d}(x)\\
    \boldsymbol{y}&=\boldsymbol{h}(\boldsymbol{x})
  \end{aligned}
  \label{system}
\end{equation}

其中,$\boldsymbol{f}: \mathbb{R}^{n}\to\mathbb{R}^{n}$ 和 $\boldsymbol{h}: \mathbb{R}^{n}\to\mathbb{R}^{p}$ 是光滑的向量场。$\boldsymbol{G}$ 是一个光滑函数,将 $\mathbb{R}^{n}\to\mathbb{R}^{n\times m}$ 映射,其中的列是光滑的向量场。$\boldsymbol{d}: \mathbb{R}^{n}\to\mathbb{R}^{n}$ 是外部扰动向量。$\boldsymbol{y}\in\mathbb{R}^{P}$ 是被控制的输出向量,可以是系统可观测输出的任意子集的函数

设 $h$ 的元素为 $h_i, i = 1, 2, ..., p$,矩阵 $G$ 的列向量为 $g_j, j = 1, 2, ..., m$,则 $h_i$ 对向量场 $f$ 和 $g_j$ 的 Lie 导数定义为
\begin{equation}
  \mathcal{L}_{f}h_{i}=\frac{\partial h_{i}}{\partial x}f,\quad\mathcal{L}_{g_{j}}h_{i}=\frac{\partial h_{i}}{\partial x}g_{j},\quad\mathcal{L}_{f}^{k}h_{i}=\frac{\partial(\mathcal{L}_{f}^{k-1}h_{i})}{\partial x}f,\quad\mathcal{L}_{g_{j}}\mathcal{L}_{f}^{k}h_{i}=\frac{\partial(\mathcal{L}_{f}^{k}h_{i})}{\partial x}g_{j}
\end{equation}

以下两项假设是所有讨论的前提:

\begin{assumption}\label{assumption1}
  系统形式为

\end{assumption}
 
\begin{assumption}\label{assumption2}
  对于每个 
\end{assumption}

\begin{remark}
  矩阵
\end{remark}

\begin{remark}
  假设 2 是为了
\end{remark}

在假设 \ref{assumption1} 下

\begin{equation}
	\left.\left[\begin{array}{c}y_1^{(\gamma_1)}\\y_2^{(\gamma_2)}\\\vdots\\y_p^{(\gamma_p)}\end{array}\right.\right]=\left[\begin{array}{cccc}\mathcal{L}_f^{\gamma_1}h_1(x)\\\\\mathcal{L}_f^{\gamma_2}h_2(x)\\\vdots\\\mathcal{L}_f^{\gamma_p}h_p(x)\end{array}\right] +  \beta(x)  \boldsymbol{u}
  \end{equation}
  或者
  \begin{equation}
	y^{(\gamma)}=\alpha(x) + \beta(x) \boldsymbol{u}
	\label{outputdynamic}
  \end{equation}
  并且行向量 $dh_{1}(x),\ldots,dL_{f}^{\gamma_{1}-1}h_{1}(x),\ldots,dh_{p}(x),\ldots,dL_{f}^{\gamma_{p}-1}h_{p}(x)$ 是线性无关的。那么系统的相对度 $\gamma$ 满足 $\gamma=\|\gamma\|_{1}=\sum_{i=1}^{p}\gamma_{i}\leq n$,并且 $h_i(x),L_{f}h_i(x),\ldots,L_{f}^{\gamma_{i}-1}h_i(x)$,$1\leq i\leq p$ 可以作为新坐标的一部分。对于 $1\leq i\leq p$,设 
  \begin{equation}
	\begin{aligned}
	  \phi_{1}^{i}(x)&= h_{i}(x)\\
	  \phi_{2}^{i}(x)&= L_{f}h_{i}(x)\\
	  &\vdots\\
	  \phi_{\gamma_{i}}^{i}(x)&= L_{f}^{\gamma_{i}-1}h_{i}(x) 
	\end{aligned}
  \end{equation}
  如果 $\gamma  = n$,对于每个 $\boldsymbol{x} \in  \mathbb{D}$,存在一个邻域 $\mathbb{N}$,使得映射 
  \begin{equation}
  \Phi=[\phi_{1}^{1}(x),\ldots,\phi_{\gamma_1}^{1}(x),\ldots,\phi_{1}^{p}(x),\ldots,\phi_{\gamma_{p}}^{p}(x)]^{T}
  \end{equation}
  在 $\mathbb{N}$ 上是一个微分同胚。否则,如果 $\gamma < n$,对于每个 $\boldsymbol{x} \in  \mathbb{D}$,总可以找到光滑函数 $\phi_{r+1}(x),\ldots,\phi_{n}(x)$,使得
  \begin{equation}
	\Phi=[\phi_{1}^{1}(x),\ldots,\phi_{\gamma_1}^{1}(x),\ldots,\phi_{1}^{p}(x),\ldots,\phi_{\gamma_{p}}^{p}(x),\phi_{\gamma+1}(x),\ldots,\phi_{n}(x)]^{T}
  \end{equation}
  在 $\boldsymbol{x}$ 的邻域 $\mathbb{N}$ 上是一个微分同胚。根据假设 \ref{assumption2},根据 Frobenius 定理,总是可以选择 $\phi_{r+1}(x),\ldots,\phi_{n}(x)$,使得对所有 $r+1\leq i\leq n$ 和所有 $1\leq j\leq m$,有 $L_{g_{j}}\phi_{i}(x)=0$,对于所有 $x \in \mathbb{N}$。 
  
  总之,如果假设 \ref{assumption1} 和 \ref{assumption2}(对于 $\gamma < n$)成立,并且忽略外部干扰,总可以找到适当的局部坐标变换 $\phi(x)$,在其下,原始系统 \eqref{system} 可以表示为标准形式。设 
  \begin{equation}
	\xi^i=\begin{pmatrix}\xi_1^i\\\xi_2^i\\\vdots\\\xi_{\gamma_i}^i\end{pmatrix}=\begin{pmatrix}\phi_1^i(x)\\\phi_2^i(x)\\\vdots\\\phi_{\gamma_i}^i(x)\end{pmatrix} \quad  
	  \xi=\begin{pmatrix}\xi^1 \\ \vdots \\ \xi^m\end{pmatrix} \quad  
	  \eta=\begin{pmatrix}\eta_1\\\eta_2\\\vdots\\\eta_{n-\gamma}\end{pmatrix}=\begin{pmatrix}\phi_{\gamma+1}(x)\\\phi_{\gamma+2}(x)\\\vdots\\\phi_n(x)\end{pmatrix} \quad   
	  z=\Phi(x)=\begin{pmatrix}\xi\\\eta\end{pmatrix}
	  \label{stran}
  \end{equation}
  然后,方程 \eqref{system} 可以重写为标准形式,为简便起见,我们仍然保留原始坐标,并以向量形式重写系统 \eqref{system} 的标准形式:
  \begin{equation}
	\begin{aligned}&\dot{\eta}= f_{0}(\eta,\xi)\\&\dot{\xi}= A_{c}\xi+B_{c}[\alpha(x)+\beta(x)u]\\&\text{y}= C_{c}\xi\end{aligned}
  \end{equation}
  % \section{增量非线性动态逆的重表述}
  其中 
  \begin{equation}
	A_c=\mathrm{diag}(A_{1},\ldots,A_{m})\quad B_c=\mathrm{diag}(B_{1},\ldots,B_{m})\quad C_c=\mathrm{diag}(C_{1},\ldots,C_{m})
  \end{equation}
  和
  \begin{equation}
	A_i=\left[\begin{array}{ccccc}0&1&0&\dots&0\\0&0&1&\dots&0\\\vdots&&\ddots&&\vdots\\\vdots&&&0&1\\0&\dots&\dots&0&0\end{array}\right]_{\gamma_i \times \gamma_i}, B_i=\left[\begin{array}{c}0\\0\\\vdots\\0\\1\end{array}\right]_{\gamma_i \times 1}, C_i=\left[\begin{array}{ccccc}1&0&\dots&0&0\end{array}\right]_{1 \times \gamma_i}
  \end{equation}
   
  如果考虑外部干扰 $d(x)$,通过坐标变换 \eqref{stran},系统 \eqref{system} 变为 
  \begin{equation}
	\begin{aligned}
	  &\dot{\eta}=\frac{\partial\eta}{\partial x}(f(x)+d(x))\Big|_{x=\Phi^{-1}(z)}= f_{d}(\eta,\xi,d)\\
	  &\dot{\xi}= A_{c}\xi+B_{c}[\alpha(x)+\beta(x)u] + L\\
	  &\text{y}= C_{c}\xi\end{aligned}
	  \label{with_external}
  \end{equation}
  其中
  \begin{equation}
	L=\begin{pmatrix}L^1\\L^2\\\vdots\\L^m\end{pmatrix} \quad \quad \quad L^i=\begin{pmatrix}L_1^i\\L_2^i\\\vdots\\L_{\gamma_i}^i\end{pmatrix}=\begin{pmatrix}L_{d}h_{i}(x)\\L_{d}L_{f}h_{i}(x)\\\vdots\\L_{d}L_{f}^{\gamma_i -1}h_{i}(x)\end{pmatrix} 
  \end{equation}

  
  