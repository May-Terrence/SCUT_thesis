\chapter{涵道风扇式无人机的姿态控制}

基于第二章建立的DFUAV非线性系统动态方程分析,其多源力与力矩耦合叠加效应导致姿态动态模型尤为复杂。如横滚角的变化会在俯仰通道和偏航通道上都产生或者间接产生力矩影响,并且三维力矩都与环境风速呈现强相关性。最常用的姿态控制方法基于反馈线性化,比如不考虑系统的数学模型的PID控制算法,虽然该方法面对非线性系统有一定的鲁棒性,但仅通过比例-积分-微分的线性组合应对内外因素产生的力矩影响难免会使系统状态出现超调和滞后等问题。此外,可以使用基于系统模型的方法考虑内外因素对系统状态的影响,但在系统模型参数过多并且不精确的情况下也将为控制算法的设计和系统调试带来很大工作量。基于上述分析,姿态控制算法设计的核心在于如何对难以测得的力矩进行在线补偿。

本章安排如下:将首先对DFUAV的姿态模型进行简化处理,方便下文姿态控制算法的设计。然后简要概述增量控制和控制分配的理论以及二者的结合,继而给出了基于INDI和优先级控制分配的DFUAV的姿态控制方案设计。为验证方案有效性,最后进行了仿真实验和飞行实验验证。

\section{增量非线性动态逆控制与控制分配理论}

NDI(非线性动态逆)是一种基于模型的非线性控制方法,其核心在于在掌握系统模型的情况下,将非线性系统在工作点处进行反馈线性化,然后通过求逆运算来消除系统中的非线性项。当系统模型参数不精确或者对系统机理认识模糊的情况下,NDI方法的控制效果将表现不佳。而INDI(增量式非线性动态逆)相比于NDI采用了增量式的控制策略,仅要求对系统的输入通道建模。INDI在未能充分掌握系统模型的情况下,通过增量式的求逆策略来实时补偿未知扰动与建模误差,得到线性化的系统动力学,进而可以通过常规的线性系统的控制方法来控制\cite{wangStabilityAnalysisIncremental2019b}。因此INDI方法对系统模型的依赖程度较小,是一种基于传感器的方法。此外,INDI方法在设计控制律时主要关注系统的输入输出特性,无需考虑应如何将控制律计算出的控制量分配到各个执行机构上。而后者所描述的任务将由控制分配算法来实现,实现过程中也无需考虑系统内部的复杂非线性因素与动态干扰。因此,采用INDI控制方法有助于将姿态控制算法设计与控制分配算法分离开,实现分层控制。

本节首先将从一般性的系统推导INDI是如何实现增量控制的,然后介绍控制分配理论以及二者的结合。

\subsection{增量非线性动态逆}

考虑如下普通的非线性动力系统:
\begin{align}
    \dot{\boldsymbol{x}}&=\boldsymbol{f}(\boldsymbol{x})+\boldsymbol{g}(\boldsymbol{x},\boldsymbol{u}) \label{3-1}
    \\
    \boldsymbol{y}&=\boldsymbol{h}(\boldsymbol{x})\label{3-2}
\end{align}
上式中$\boldsymbol{x}\in\mathbb{R}^{n}$为n维系统状态变量,$\boldsymbol{u}\in\boldsymbol{U}\in\mathbb{R}^{p}$为p维系统输入变量,其中$\boldsymbol{U}$是容许控制集合,表示对输入变量的约束,$\boldsymbol{y}\in\mathbb{R}^{m}$为m维系统输出变量。映射关系$\boldsymbol{f}:\mathbb{R}^n\rightarrow\mathbb{R}^n$,$\boldsymbol{g}:\mathbb{R}^n\times\mathbb{R}^p\rightarrow\mathbb{R}^n$,$\boldsymbol{h}:\mathbb{R}^n\rightarrow\mathbb{R}^m$。

式\eqref{3-1}中系统状态的一阶泰勒展开式为:
\begin{equation}
    \dot{\boldsymbol{x}}=\dot{\boldsymbol{x}}_0+f^{\prime}(\boldsymbol{x})(\boldsymbol{x}-\boldsymbol{x}_0)+\left.\frac{\partial \boldsymbol{g}}{\partial \boldsymbol{x}}\right|_{\boldsymbol{x}=\boldsymbol{x}_0}(\boldsymbol{x}-\boldsymbol{x}_0)+\left.\frac{\partial \boldsymbol{g}}{\partial \boldsymbol{u}}\right|_{\boldsymbol{u}=\boldsymbol{u}_0}(\boldsymbol{u}-\boldsymbol{u}_0)
    \label{3-3}
\end{equation}
然后对式\eqref{3-2}中的输出表达式求一次导数,结合复合函数求导的链式法则与式\eqref{3-3},可得到动力系统输入与输出之间的关系:
\begin{equation}
    \begin{aligned}
    \dot{\boldsymbol{y}} = \underbrace{\left.\frac{\partial\boldsymbol{h}}{\partial\boldsymbol{x}}\right|_{\boldsymbol{x}=\boldsymbol{x}_0}\dot{\boldsymbol{x}}_0}_{\dot{\boldsymbol{y}}_0} + \left.\frac{\partial \boldsymbol{h}}{\partial \boldsymbol{x}}\right|_{\boldsymbol{x}=\boldsymbol{x}_0}\begin{bmatrix}f^{\prime}(\boldsymbol{x})(\boldsymbol{x}-\boldsymbol{x}_0)+\left.\dfrac{\partial \boldsymbol{g}}{\partial \boldsymbol{x}}\right|_{\boldsymbol{x}=\boldsymbol{x}_0}(\boldsymbol{x}-\boldsymbol{x}_0)+\left.\dfrac{\partial \boldsymbol{g}}{\partial \boldsymbol{u}}\right|_{\boldsymbol{u}=\boldsymbol{u}_0}(\boldsymbol{u}-\boldsymbol{u}_0)\end{bmatrix}
    \end{aligned}
    \label{3-4}
\end{equation}

INDI理论的一个核心假设是时间尺度分离法则\cite{inproceedings},该法则基于系统动力学的时间尺度特性,指出系统内部状态变量的动态响应速率要明显慢于外部输入信号的时变特性。基于这一动力学特性差异,相比于外部输入信号的导数项,系统状态变量的导数项可以视为高阶小量并忽略不计,从而有效地简化系统的动态方程。

基于上述核心假设,可以将式\eqref{3-4}中系统状态的导数项$(\boldsymbol{x}-\boldsymbol{x}_0)$忽略,得到:
\begin{equation}
    \begin{aligned}
    \dot{\boldsymbol{y}} &= \dot{\boldsymbol{y}}_0 +  \left.\frac{\partial \boldsymbol{h}}{\partial \boldsymbol{x}}\right|_{\boldsymbol{x}=\boldsymbol{x}_0}\left.\dfrac{\partial \boldsymbol{g}}{\partial \boldsymbol{u}}\right|_{\boldsymbol{u}=\boldsymbol{u}_0}(\boldsymbol{u}-\boldsymbol{u}_0)\\
        &= \dot{\boldsymbol{y}}_0 + \boldsymbol{G}(\boldsymbol{u}-\boldsymbol{u}_0)
    \end{aligned}
    \label{3-5}
\end{equation}
其中$\dot{\boldsymbol{y}}_0$为系统当前时刻输出的导数,$\boldsymbol{G}\in\mathbb{R}^{m\times p}$,$\boldsymbol{u}_0$为系统当前时刻的控制输入。
使用下标$(.)_{d}$表示该变量的期望值,那么根据公式\eqref{3-5},可以得到期望的输入变量$\boldsymbol{u}_{d}$表示为:
\begin{equation}
    \begin{aligned}
    \boldsymbol{u}_{d}&=\boldsymbol{u}_{0}+\Delta u\\
    &=\boldsymbol{u}_{0}+\boldsymbol{G}^{-1}(\dot{\boldsymbol{y}_d}-\dot{\boldsymbol{y}})
    \label{3-6}
    \end{aligned}
\end{equation}
新的控制输入是在当前时刻的控制输入基础上加上一个控制增量$\Delta u$得到。通过对控制效率矩阵求逆,并且找到期望的输出导数与当前时刻的输出导数的误差,可以得到控制增量。

\subsection{控制分配理论}

继续考虑3.1.1小节中引入的非线性动力系统。对于一个过驱动系统,其系统输入的维度大于系统输出的维度,即$p>m$,表明执行机构存在冗余。将$\boldsymbol{g}(\boldsymbol{x},\boldsymbol{u})$做矩阵分解得到:
\begin{equation}
    \boldsymbol{g}(\boldsymbol{x},\boldsymbol{u})=\boldsymbol{D}\boldsymbol{K}(\boldsymbol{x},\boldsymbol{u})
    \label{3-7}
\end{equation}
其中$\boldsymbol{D}\in\mathbb{R}^{n\times m}$,映射关系$\boldsymbol{K}:\mathbb{R}^n\times\mathbb{R}^p\rightarrow\mathbb{R}^m$。

引入$m$维虚拟控制输入$\boldsymbol{\nu}$:
\begin{equation}
    \boldsymbol{\nu}=\boldsymbol{K}(\boldsymbol{x},\boldsymbol{u})
    \label{3-8}
\end{equation}
由于实际控制输入的约束$\boldsymbol{U}$存在,将实际控制输入变换为虚拟控制输入后有$\boldsymbol{\nu}\in\boldsymbol{\Phi}\in\mathbb{R}^{m}$。容许控制集合通过映射$\boldsymbol{K}$从$\mathbb{R}^{p}$空间映射到$\mathbb{R}^{m}$空间得到可达控制集合,如果$\boldsymbol{\nu}\in\boldsymbol{\Phi}$,则称虚拟控制输入为可达,否则为不可达。

对$\boldsymbol{\nu}$做一阶泰勒展开并且同样忽略关于系统状态的导数项,得到:
\begin{equation}
    \begin{aligned}
    \boldsymbol{\nu} &= \boldsymbol{\nu}_0 +  \left.\dfrac{\partial \boldsymbol{K}}{\partial \boldsymbol{u}}\right|_{\boldsymbol{u}=\boldsymbol{u}_0}(\boldsymbol{u}-\boldsymbol{u}_0)\\
        &= \boldsymbol{\nu}_0 + \boldsymbol{B}(\boldsymbol{u}-\boldsymbol{u}_0)
    \end{aligned}
    \label{3-9}
\end{equation}
其中$\boldsymbol{\nu}_0$表示系统当前时刻的虚拟控制输入,$\boldsymbol{B}\in\mathbb{R}^{m\times p}$。将$(\boldsymbol{u}-\boldsymbol{u}_0)$代入公式\eqref{3-5},得到:
\begin{equation}
    \begin{aligned}
    \dot{\boldsymbol{y}}=\dot{\boldsymbol{y}}_0+\boldsymbol{G}\boldsymbol{B}^\dagger(\boldsymbol{\nu}-\boldsymbol{\nu}_0)\\
    \Rightarrow \dot{\boldsymbol{y}_d}-\dot{\boldsymbol{y}}_0 = \boldsymbol{G}\boldsymbol{B}^\dagger\Delta\boldsymbol{\nu}
    \label{3-10}
    \end{aligned}
\end{equation}
其中$\boldsymbol{B}^\dagger$表示矩阵$\boldsymbol{B}$的伪逆矩阵,有以下关系:
\begin{gather}
    \boldsymbol{u}=\boldsymbol{B}^\dagger\boldsymbol{\nu}    \label{3-11}
    \\ 
    \boldsymbol{B}^\dagger=\boldsymbol{B}^T\left(\boldsymbol{B}\boldsymbol{B}^T\right)^{-1}    \label{3-12}
\end{gather}

公式\eqref{3-11}中直接表示虚拟控制输入$\boldsymbol{\nu}$与实际控制输入$\boldsymbol{u}$之间映射关系的方法就是伪逆法。在分层控制架构中,系统采用上层决策-底层执行的层级化设计模式:上层控制算法承担决策功能,专注于在线求解系统动力学方程生成期望控制指令,避免了执行机构物理约束对优化问题的干扰,降低了计算复杂度;底层控制分配算法则负责执行任务,采用优化计算方法将上层指令解析为各执行机构的物理控制量,实现控制指令在冗余执行机构中的最优分配。文献\parencite{harkegardResolvingActuatorRedundancy2005}中讨论了层级化设计方法和全阶最优控制设计的联系。采用独立的控制分配算法,优势之一是可以考虑执行机构的物理约束,比如当某个执行机构饱和时,在条件允许的情况下其余执行机构可以用于弥补因该执行机构饱和而损失的控制效能 。图\ref{理论框图}展示了基于INDI的上层控制算法和基于伪逆法的底层控制分配算法的控制系统框图。

\begin{figure}[htbp]
	\centering
	\begin{minipage}[c]{1\textwidth}
		\centering
		\includegraphics[scale=1]{Fig/理论框图.pdf}
		\caption{\label{理论框图}基于INDI和伪逆法的控制系统框图}
	\end{minipage}%
\end{figure}

需特别指出的是,尽管可以设计出渐进稳定的控制算法,但是受矩阵$\boldsymbol{B}$奇异特性及执行机构物理约束的影响,所推导的期望虚拟控制输入$\boldsymbol{\nu}_d$可能属于不可达集。这一现象在本质上源于伪逆法在控制分配问题中的固有局限性:当控制指令超出执行机构可达包络时,将导致控制分配问题退化为无可行解的数学约束条件,因此下文在控制分配算法的设计中将参考文献\parencite{HKXB202010026},采用优先级控制分配方法。该方法旨在通过优先级排序的方式,优先满足高优先级的控制输入分量以达到更大范围的容许控制。

\section{姿态模型简化}

根据第二章的建模总结可知,DFUAV的姿态模型涉及到多个系统状态变量间的耦合并且与许多模型参数和环境风速有关,尤其是角速度动态方程\eqref{eq_43}中涉及到的多源力矩模型。为便于下文姿态控制算法的设计,本节将对所总结的力矩模型进行合理且适当的简化,根据现有条件区分可控力矩与不可控力矩,然后分别处理。

根据式\ref{eq_11}可知,DFUAV在机体坐标系下的合外力矩可以分解为:
\begin{equation}
    \boldsymbol{M}^b=\boldsymbol{M}_{fan}^b+\boldsymbol{M}_{aero}^b+\boldsymbol{M}_{duct}^b+\boldsymbol{M}_{vane}^b+\boldsymbol{M}_{gyro}^b+\boldsymbol{M}_{flap}^b
    \label{3-13}
\end{equation}
由于涵道风扇的转速$\Omega$可由电调读取,机体的角速度$\boldsymbol\omega^b$可由机载的IMU测量得到,结合表\ref{DFUAV_parameters}中的模型参数,可以计算出涵道风扇的扭矩$\boldsymbol{M}_{fan}^b$和陀螺力矩$\boldsymbol{M}_{gyro}^b$。

模型假设环境风速在$\boldsymbol{O}_b-\boldsymbol{Z}_b$轴的分量为零,即$V_0=0$。根据式\ref{eq_20}以及式\ref{eq_22},可将涵道出口风速$V_{e}$简化为:
\begin{equation}
    V_e=\sqrt{\frac{k_{fan}}{\sigma_d\rho S}}\Omega={k_{f}}\Omega    \label{3-14}
\end{equation}
在这种假设下,涵道出口风速$V_{e}$与风扇转速$\Omega$之间呈正比关系。不妨设$V_e=k_{f}\Omega$,其中$k_{f}=\sqrt{\frac{k_{fan}}{\sigma_d\rho S}}$可以根据表\ref{DFUAV_parameters}中的参数计算得到。然后根据控制舵面的偏转角$\delta_{i}$和涵道力臂的长度,可计算出控制舵面产生的力矩$\boldsymbol{M}_{vane}^b$。

对于在固定气动面上产生的反扭距$\boldsymbol{M}_{flap}^b$,由于$V_{e}$与$\Omega$为正比关系,可将式\ref{eq_39}简化为:
\begin{equation}
    \boldsymbol{M}_{flap}^b=
    \begin{bmatrix}
    0 \\0 \\k_{flap}\Omega^2
    \end{bmatrix}
    \label{3-15}
\end{equation}

现在考虑风扇扭矩$\boldsymbol{M}_{fan}^b$、固定气动面上的反扭距$\boldsymbol{M}_{flap}^b$和控制舵面在$\boldsymbol{O}_b-\boldsymbol{Z}_b$轴的偏航力矩。根据第二章的力矩分析,仅有上述三部分分量会对机体的偏航力矩产生影响。为使机体的偏航角保持稳定,有如下关系:
\begin{equation}
    \begin{gathered}
    -k_q\Omega^2+k_{flap}\Omega^2+k_{\delta}k_f^2 l_2(\delta_1 + \delta_2 + \delta_3 + \delta_4)\Omega^2=0
     \\
     \Rightarrow
    -k_q+k_{flap}+k_{\delta}k_f^2 l_2(\delta_1 + \delta_2 + \delta_3 + \delta_4)=0
    \end{gathered}
    \label{3-16}
\end{equation}
式\ref{3-15}中等式约去风扇转速变量$\Omega$后,只有控制舵面的偏转角是变量,其他均为确定的模型参数。这表明了当机体偏航角稳定时,所需的控制舵面偏转角是一定的。由于$k_{flap}$与固定气动面的设计形状有关,不失一般性,本文假设固定气动面在恰当的设计下可以使得$k_{flap}=k_{q}$,即:
\begin{equation}
    \boldsymbol{M}_{flap}^b + \boldsymbol{M}_{fan}^b = \boldsymbol{0}
\label{3-17}
\end{equation}

对于在机身上产生的气动力矩$\boldsymbol{M}_{aero}^b$和由于涵道翼型产生的力矩$\boldsymbol{M}_{duct}^b$,由于机体相对于气流的速度$\boldsymbol{V}_a^b$在现有条件下是未知的,所以将其视为不可控力矩来作为未建模动态。使用$\boldsymbol{M}_{a}^b$统一表示:
\begin{equation}
    \boldsymbol{M}_{a}^b=\boldsymbol{M}_{aero}^b+\boldsymbol{M}_{duct}^b
    \label{3-18}
\end{equation}

$\boldsymbol{M}_{a}^b$可以描述为地面坐标系下的速度$\boldsymbol{V}^e$、姿态$\boldsymbol{\eta}$和环境风速$\boldsymbol{W}^e$的函数,即:
\begin{equation}
    \boldsymbol{M}_{a}^b=\boldsymbol{M}_{a}^b(\boldsymbol{V}^e,\boldsymbol{\eta},\boldsymbol{W}^e)
    \label{3-19}
\end{equation}

综合以上分析,简化后的气动力矩模型可以写为:
\begin{equation}
    \boldsymbol{M}^b=\boldsymbol{M}_{vane}^b+\boldsymbol{M}_{gyro}^b+\boldsymbol{M}_{a}^b
    \label{3-20}
\end{equation}
与力矩模型相关的角速度动态方程对应的可以简写为:
\begin{equation}
    \begin{aligned}
        \dot{p}  &=\frac{1}{J_x}\Bigg\{ (J_y-J_z)qr+
        \big[- l_1k_\delta V_e^2(\delta_1 - \delta_3)-J_{fan}\Omega{q} +\boldsymbol{M}_{ax}^b\big]\Bigg\} \\
        \dot{q}  &=\frac{1}{J_y}\Bigg\{ (J_z-J_x)pr+
        \big[l_1k_\delta V_e^2(\delta_4 - \delta_2)+J_{fan}\Omega{p} +\boldsymbol{M}_{ay}^b\big]\Bigg\} \\
        \dot{r}  &=\frac{1}{J_z}\Bigg\{ (J_x-J_y)pq+
        \left[l_2k_\delta V_e^2(\delta_1 + \delta_2 + \delta_3 + \delta_4) + \boldsymbol{M}_{az}^b\right]
        \Bigg\}
    \end{aligned}
    \label{3-21}
\end{equation}
上述简化的力矩模型将可控力矩与不可控力矩分开。控制舵面合力矩$\boldsymbol{M}_{vane}^b$作为可控力矩由下文设计的姿态控制算法来计算期望力矩(即期望的控制舵面偏转角度)。陀螺力矩$\boldsymbol{M}_{gyro}^b$同样作为可控力矩,但该力矩会对DFUAV的姿态造成负面影响,下文将根据其产生的机理设计对应的陀螺力矩补偿算法加以补偿。而$\boldsymbol{M}_{a}^b$作为不可控力矩将通过姿态控制算法在在线补偿。

% \subsection{虚拟控制输入}

% 如前文所述,DFUAV姿态子系统的控制输入为控制舵面的偏转角$\delta_i$,由四片控制舵面产生三维控制力矩,因此带来了过驱动的问题。因此引入虚拟控制输入$\boldsymbol{\nu}$将姿态子系统分为两部分,$\boldsymbol{\nu}$既作为姿态子系统的控制输入也作为控制分配算法的期望力矩。可表示为:
% \begin{equation}
%     \boldsymbol{\nu}=
%     \begin{bmatrix}
%     {\nu}_x \\ {\nu}_y \\ {\nu}_z
%     \end{bmatrix}=\boldsymbol{B}
%     \begin{bmatrix}
%     \delta_1 \\ \delta_2 \\ \delta_3 \\ \delta_4
%     \end{bmatrix}
%     \label{3-20}
% \end{equation}
% 其中矩阵$\boldsymbol{B}$为四片控制舵面偏转角到三维虚拟控制输入的转换矩阵,有$\boldsymbol{B}\in\mathbb{R}^{3\times4}$。

\section{基于INDI和优先级控制分配的姿态控制方案}

根据本章开篇时的分析,姿态控制算法设计的核心在于如何对难以测得的力矩进行在线补偿。在第二节姿态模型简化中,将难以测得的力矩视为了未建模动态$\boldsymbol{M}_{a}^b$。基于INDI的控制架构设计,其优势主要体现在模型的鲁棒性与未建模误差的动态补偿能力。式\eqref{3-21}中的形式已十分适用于INDI控制器的设计:一方面,姿态子系统的输入通道的模型机理已经明确,由四片独立的控制舵面产生偏转角作为输入信号;另一方面,由于未建模动态$\boldsymbol{M}_{a}^b$作用于角加速度,所以可用机载IMU高频测量角速度并估算角加速度,并根据INDI的增量式求逆策略在线补偿未建模动态$\boldsymbol{M}_{a}^b$。在所构建的控制架构中,控制输入分为两部分:INDI输入与反馈输入。INDI输入针对未建模动态进行补偿,将非线性系统转化为线性形式;而反馈输入针对此线性形式设计反馈控制律,保证系统的稳定性和鲁棒性。

根据优先级控制分配的思想,需要将控制输入分量进行优先级排序。由于INDI输入包含对未建模动态的补偿,反馈输入使用线性的状态反馈设计,与系统响应相关。系统能够稳定运行的前提是首先消除未建模动态的影响,将非线性系统化为线性系统,然后再设计反馈控制律,确保系统的稳定性与响应能力。因此本文中将INDI输入视为高优先级分量,反馈输入视为低优先级分量。确保INDI输入不会产生分配误差,在控制舵面受到物理约束的情况下仅在反馈输入中产生分配误差。具体而言,这种划分方法可以在一定程度上避免系统输出耦合现象\cite{HKXB202010026}。

\subsection{INDI姿态控制架构}

对于大部分无人机姿态运动控制,常采用基于欧拉角参数化的局部线性化方法,横滚角、俯仰角和偏航角三个通道分别作为控制目标单独控制,使得无人机的旋转过程被线性化处理。但是不同于位置动态方程\eqref{eq_40},欧拉角动态方程\eqref{eq_42}将三个姿态通道耦合在一起。实际上,无人机的姿态属于$SO(3)$上的流形结构,任何姿态变化都被视为流形上的测地线运动。而传统的局部线性化跟踪姿态的方法相当于在流形的切空间进行局部线性逼近,这种线性化处理方式导致了姿态轨迹生成过程中的不连续性,丢失了流形的曲率信息,所以在机动飞行时会导致姿态跟踪误差变大。因此本文参考文献\parencite{10.1115/1.4052714}在$SO(3)$上考虑姿态误差的表示。

由于旋转矩阵$\boldsymbol{R}\in{SO(3)}$,姿态欧拉角$\boldsymbol{\eta}\in\mathbb{R}^3$,因此定义符号$[.]_\times$表示映射关系$\boldsymbol{R}^3\to SO(3)$,$(.)^\vee$表示对应的逆映射$SO(3)\to\boldsymbol{R}^3$。对于任意姿态$\boldsymbol{\eta}=[\varphi \quad \theta \quad \psi]^T$,对应的矩阵形式可以表示为:
\begin{equation}
    \boldsymbol{R}=[\boldsymbol{\eta}]_\times=
    \begin{bmatrix}
    0 & -\psi & \theta \\
    \psi & 0 & -\varphi \\
    -\theta & \varphi & 0
    \end{bmatrix}
    \label{3-22}
\end{equation}

假设期望姿态为$\boldsymbol{\eta}_d$,当前姿态为$\boldsymbol{\eta}$,对应的矩阵形式分别为$\boldsymbol{R}_d$和$\boldsymbol{R}$。则在${SO(3)}$上的姿态误差可以表示为:
\begin{equation}
    \boldsymbol{e}_R=\dfrac{1}{2}(\boldsymbol{R}_d^T\boldsymbol{R}-\boldsymbol{R}^T\boldsymbol{R}_d)^\vee\in \mathbb{R}^3
    \label{3-23}
\end{equation}

因此期望的姿态欧拉角变化率可以表示为:
\begin{equation}
    \dot{\boldsymbol{\eta}}_1=-\boldsymbol{K}_R\boldsymbol{e}_R
    \label{3-24}
\end{equation}
其中$\boldsymbol{K}_R$为姿态误差的正定增益矩阵。式\eqref{3-24}所示的作差方式类似于于反馈信号减去给定信号,所以需要加负号。

为提高系统的响应速度,确保快速跟踪期望姿态,引入微分前馈策略:
\begin{equation}
    \dot{\boldsymbol{\eta}}_2=\frac{d\boldsymbol{\eta}_d}{dt}
    \label{3-25}
\end{equation}

结合式\eqref{eq_10}、式\eqref{3-24}和式\eqref{3-25},期望的机体角速度$\boldsymbol\omega^{b}_d$可以表示为:
\begin{equation}
    \boldsymbol{\omega}^{b}_d=\boldsymbol{Q}^{-1}( \dot{\boldsymbol{\eta}}_1+ \dot{\boldsymbol{\eta}}_2)
    \label{3-26}
\end{equation}

接下来考虑与动力学相关的角速度动态,包含了可控力矩与不可控力矩。如前文所述,DFUAV姿态子系统的控制输入为四片相互独立的控制舵面的偏转角$\delta_i$,由四个偏转角度的组合产生用于控制机体姿态的俯仰力矩、滚转力矩和偏航力矩,因此姿态子系统中存在执行机构的冗余,存在过驱动的问题。引入虚拟控制输入$\boldsymbol{\nu}\in\mathbb{R}^3$将姿态控制算法与控制分配算法分开:
\begin{equation}
    \boldsymbol{\nu}=
    \begin{bmatrix}
    {\nu}_x \\ {\nu}_y \\ {\nu}_z
    \end{bmatrix}=\boldsymbol{B}
    \begin{bmatrix}
    \delta_1 \\ \delta_2 \\ \delta_3 \\ \delta_4
    \end{bmatrix}
    \label{3-27}
\end{equation}
其中矩阵$\boldsymbol{B}$表示四片独立控制舵面偏转角到三维虚拟控制输入的映射矩阵,有$\boldsymbol{B}\in\mathbb{R}^{3\times4}$。

结合式\eqref{eq_10}、式\eqref{eq_36}、式\eqref{3-14}和式\eqref{3-27},可以将可控的控制舵面力矩重写为模块化形式:
\begin{equation}
    (\boldsymbol{J}^b)^{-1}\boldsymbol{M}_{vane}^b=\boldsymbol{H}_{vane}\boldsymbol{\nu}\Omega^2
    \label{3-28}
\end{equation}
其中
\begin{gather}
\left.\boldsymbol{H}_{vane}\triangleq k_{\delta}k_f^2\left[
    \begin{array}
    {ccc}2J_x^{-1}l_1 & & \\
        & 2J_y^{-1}l_1 & \\
        & & 4J_z^{-1}l_2
    \end{array}\right.\right]\in\mathbb{R}^{3\times3}
    \label{3-29}\\
    \boldsymbol{B}\triangleq
    \begin{bmatrix}
    -0.5 & 0 & 0.5 & 0 \\
    0 & -0.5 & 0 & 0.5 \\
    0.25 & 0.25 & 0.25 & 0.25
    \end{bmatrix}
    \label{3-30}
\end{gather}
经过式\eqref{3-28}的模块化处理,由原本的从控制舵面角度到控制力矩的映射转变为了从控制舵面角度到虚拟控制输入的映射。在此过程中,映射矩阵剔除了所有与模型相关参数项。

对于可控的陀螺力矩$\boldsymbol{M}_{gyro}^b$,根据式\eqref{eq_10}和式\ref{eq_38},可将其模块化表示为:
\begin{equation}
    (\boldsymbol{J}^b)^{-1}\boldsymbol{M}_{gyro}^b=\boldsymbol{H}_{gyro}(\boldsymbol{\omega}^b)\Omega
    \label{3-31}
\end{equation}
其中
\begin{equation}
    \boldsymbol{H}_{gyro}(\boldsymbol{\omega}^b) \triangleq J_{fan}[-J_x^{-1}q\quad J_y^{-1}p\quad0]^T\in\mathbb{R}^{3}
    \label{3-32}
\end{equation}

对于不可控的气动力矩$\boldsymbol{M}_{a}^b$,将其与式\eqref{eq_10}中的非线性项$(\boldsymbol{J}^b\boldsymbol{\omega}^b\times\boldsymbol{\omega}^b)$一同处理,表示为与地面坐标系下的速度$\boldsymbol{V}^e$、姿态$\boldsymbol{\eta}$、环境风速$\boldsymbol{W}^e$和机体角速度$\boldsymbol{\omega}^b$有关的函数:
\begin{equation}
    ({\boldsymbol{J}^b})^{-1}(\boldsymbol{M}_{a}^b+\boldsymbol{J}\boldsymbol{\omega}^b\times\boldsymbol{\omega}^b)\triangleq \boldsymbol{L}(\boldsymbol{V}^e,\boldsymbol{\eta},\boldsymbol{W}^e,\boldsymbol{\omega}^b)\in\mathbb{R}^3
    \label{3-33}
\end{equation}

结合式\eqref{3-28}、式\eqref{3-31}和式\eqref{3-33},角速度动态方程可以被模块化表示为:
\begin{equation}
    \boldsymbol{\dot{\omega}}^b=\boldsymbol{H}_{vane}\boldsymbol{\nu}\Omega^2+\boldsymbol{H}_{gyro}(\boldsymbol{\omega}^b)\Omega+\boldsymbol{L}(\boldsymbol{V}^e,\boldsymbol{\eta},\boldsymbol{W}^e,\boldsymbol{\omega}^b)
    \label{3-34}
\end{equation}
可以直接在该模型的基础上应用INDI,而无需考虑控制分配问题。根据本章第一节介绍的INDI理论,为了得到$\boldsymbol{\omega}^b$的增量表达式,对公式\eqref{3-34}在其工作点处(使用下标$(.)_0$表示)作一阶泰勒展开:
\begin{equation}
    \begin{aligned}
    \boldsymbol{\dot{\omega}}^b&=\boldsymbol{H}_{vane}\boldsymbol{\nu}\Omega^2+\boldsymbol{H}_{gyro}(\boldsymbol{\omega}^b)\Omega+\boldsymbol{L}(\boldsymbol{V}^e,\boldsymbol{\eta},\boldsymbol{W}^e,\boldsymbol{\omega}^b)\\
    &=\boldsymbol{H}_{vane}\boldsymbol{\nu}_0\Omega_0^2+\boldsymbol{H}_{gyro}(\boldsymbol{\omega}_0^b)\Omega_0+\boldsymbol{L}_0\\
    & +\left.\frac{\partial(\boldsymbol{H}_{vane}\boldsymbol{\nu}\Omega^2)}{\partial\boldsymbol{\nu}}\right|_{\boldsymbol{\nu}=\boldsymbol{\nu}_0}(\boldsymbol{\nu}-\boldsymbol{\nu}_0)
    +\left.\bigg[\frac{\partial(\boldsymbol{H}_{vane}\boldsymbol{\nu}\Omega^2)}{\partial\Omega}+\frac{\partial(\boldsymbol{H}_{gyro}\Omega)}{\partial\Omega}\bigg]\right|_{\Omega=\Omega_0}
    (\Omega-\Omega_0) \\
     & +\left.\bigg[\frac{\partial(\boldsymbol{H}_{gyro}\Omega)}{\partial\boldsymbol{\omega}^{b}}+\frac{\partial\boldsymbol{L}}{\partial\boldsymbol{\omega}^{b}}\bigg]\right|_{\boldsymbol{\omega}^{b}=\boldsymbol{\omega}_{0}^{b}}(\boldsymbol{\omega}^{b}-\boldsymbol{\omega}_{0}^{b})\\
    & +\left.\frac{\partial\boldsymbol{L}}{\partial\boldsymbol{V}^{e}}\right|_{\boldsymbol{V}^{e}=\boldsymbol{V}_{0}^{e}}
    (\boldsymbol{V}^{e}-\boldsymbol{V}_{0}^{e})
    +\left.\frac{\partial\boldsymbol{L}}{\partial\boldsymbol{\eta}}\right|_{\boldsymbol{\eta}=\boldsymbol{\eta}_{0}}(\boldsymbol{\eta}-\boldsymbol{\eta}_{0})
    +\left. \frac{\partial\boldsymbol{L}}{\partial\boldsymbol{W}^e}\right|_{\boldsymbol{W}^e=\boldsymbol{W}_0{}^e}(\boldsymbol{W}^e-\boldsymbol{W}_0{}^e)
    \end{aligned}
    \label{3-35}
\end{equation}
展开式中的第一项$\boldsymbol{H}_{vane}\boldsymbol{\nu}_0\Omega_0^2+\boldsymbol{H}_{gyro}(\boldsymbol{\omega}_0^b)\Omega_0+\boldsymbol{L}_0$等价于工作点处的角加速度,可以由机载IMU测量的角速度经过差分滤波得到:
\begin{equation}
    \boldsymbol{\dot{\omega}}^b_0=\boldsymbol{H}_{vane}\boldsymbol{\nu}_0\Omega_0^2+\boldsymbol{H}_{gyro}(\boldsymbol{\omega}_0^b)\Omega_0+\boldsymbol{L}_0
    \label{3-36}
\end{equation}

展开式中的其他项是关于系统内部状态变量$(\boldsymbol{V}^e,\boldsymbol{\eta},\boldsymbol{\omega}^b)$、系统控制输入$(\boldsymbol{\nu},\Omega)$和外部扰动$\boldsymbol{W^e}$的一阶偏导数项。根据本章第一小节介绍的时间尺度分离法则,认为在角加速度的输入信号的时间尺度内,系统内部状态变量包括速度、姿态和机体角速度的变化是缓慢的,因此可以将这些状态变量的一阶偏导数项视为0:
\begin{equation}
    \begin{cases}
        \boldsymbol{V}^e-\boldsymbol{V}_0^e\approx0 \\
        \boldsymbol{\eta}-\boldsymbol{\eta}_0\approx0 \\
        \boldsymbol{\omega}^b-\boldsymbol{\omega}_0^b\approx0
    \end{cases}
    \label{3-37}
\end{equation}

在无人机控制中,执行机构(如电机、舵机等)可在数十毫秒的时间尺度内完成控制指令的执行,产生所需的力与力矩。而机体的速度、姿态等状态变量受空气动力阻尼、质量惯性等因素的制约,其动态过程一般落后于执行机构响应1-2个数量级,因此上述假设是合理的。此外,对于外部环境风速扰动$\boldsymbol{W}^e$的变化是未知并且难以预测的,因此假定$(\boldsymbol{W}^e-\boldsymbol{W}_0^e\approx0)$。但此假设并没有排除外部环境风速扰动对系统的影响,缓慢变化的自然风对姿态造成的影响已被考虑在$\boldsymbol{L}_0$中。而应对快速变化的风速扰动则取决于所设计的姿态控制器的鲁棒性。

其余关于外部控制输入的偏导数项如下:
\begin{equation}
    \begin{gathered}
    \left.\frac{\partial(\boldsymbol{H}_{vane}\boldsymbol{\nu}\Omega^2)}{\partial\boldsymbol{\nu}}\right|_{\boldsymbol{\nu}=\boldsymbol{\nu}_0}(\boldsymbol{\nu}-\boldsymbol{\nu}_0)=\boldsymbol{H}_{vane}\Omega_0^2(\boldsymbol{\nu}-\boldsymbol{\nu}_0) \\
    \left.\bigg[\frac{\partial(\boldsymbol{H}_{vane}\boldsymbol{\nu}\Omega^2)}{\partial\Omega}+\frac{\partial(\boldsymbol{H}_{gyro}\Omega)}{\partial\Omega}\bigg]\right|_{\Omega=\Omega_0}(\Omega-\Omega_0)=[2\boldsymbol{H}_{vane}\boldsymbol{\nu}_0\Omega_0+\boldsymbol{H}_{gyro}(\boldsymbol{\omega}_0^b)](\Omega-\Omega_0)
    \end{gathered}
    \label{3-38}
\end{equation}

因此,角速度动态的一阶泰勒展开式\eqref{3-35}近似可以表示为:
\begin{equation}
    \begin{aligned}
    \boldsymbol{\dot{\omega}}^b&=\boldsymbol{\dot{\omega}}_0^b+\boldsymbol{H}_{vane}\Omega_0^2(\boldsymbol{\nu}-\boldsymbol{\nu}_0)+[2\boldsymbol{H}_{vane}\boldsymbol{\nu}_0\Omega_0+\boldsymbol{H}_{gyro}(\boldsymbol{\omega}_0^b)](\Omega-\Omega_0)\\
    \end{aligned}
    \label{3-39}
\end{equation}

值得一提的是,风扇转速$\Omega$作为速度控制的输入信号并不直接参与姿态控制,因此不被视为姿态控制中输入信号的一部分。可以将$\Omega$作为在角速度动态中的已知参数,该参数可以从速度环设计中得到。为了简便表示,采用以下记号:






\section{图}
图的导入需要提前准备好图片文件,最好是.png、.eps、.pdf或.jpg文件。另外,如果是从matlab导出图片文件,可使用print函数或手动导出,print函数的使用可参考ICGNC2020plot.m以及PlotToFileColorPDF.m文件等。手动导出(matlab的figure界面的“文件”->“导出设置”设置好大小、分辨率和线宽等然后点击“应用于图窗”)主要用于观察效果,可设置某种样式名称后保存该样式,下次使用时加载,具体可百度“matlab导出高清图片”。需要特别注意的是一定要1:1导入matlab生成的图片,并且图中文字设置好字体字号。否则缩放之后,图片的字号就变了,盲审老师一眼就能看出来字号不对,就很麻烦。这就是为什么要在matlab点击“应用于图窗“进行预览,观测效果后再1:1使用图片。

使用如下代码放置独立成行的图片,效果如图\ref{one_DFUAV}所示
\begin{lstlisting}
\begin{figure}[htbp]
	% 图片居中(列居中对齐)
	\centering	
	% 包含当前路径下的Fig文件夹的图片文件DFUAV_f31.png
	\includegraphics[scale=1]{Fig/DFUAV_f31.png} 
	% 添加标签one_DFUAV以及图标题“涵道风扇式无人机”,引用某图时使用\ref{xxx},其中xxx就是标签,图编号是自动生成的。
	\caption{\label{one_DFUAV}涵道风扇式无人机} 
\end{figure}
\end{lstlisting}
其中figure为环境名,[htbp]表示将图片设置为浮动体,实际上这在.cls文件已经设置过,因而可以省略。[scale=1]表示安装1:1的比例导入图片,还可以按其他方式导入,需要时可自行百度。
\begin{figure}[htbp]
	\centering
	\includegraphics[scale=1]{Fig/DFUAV_f31.png}
	\caption{\label{one_DFUAV}涵道风扇式无人机}
\end{figure}

使用如下代码划分页面并排放置图\ref{Hawk}、图\ref{GTSpy}
\begin{lstlisting}
\begin{figure}[htbp]
	\centering
	\begin{minipage}[c]{0.5\textwidth} % minipage将页面划分为0.5\textwidth
		\centering
		\includegraphics[width=6cm,height=6cm]{Fig/honeywell_t-hawk.jpg}
		\caption{\label{Hawk}T-Hawk}
	\end{minipage}%
	\begin{minipage}[c]{0.5\textwidth}
		\centering
		\includegraphics[width=6cm,height=6cm]{Fig/GTSpy.jpg}
		\caption{\label{GTSpy}GTSpy}
	\end{minipage}
\end{figure}
\end{lstlisting}
其中[c]表示行居中对齐。当图片大小不一但又需要1:1导入时,图标题可能行不对齐,因此可以改为如下指令:
\begin{lstlisting}
\begin{figure}[htbp]
	\centering
	\begin{minipage}[c]{0.5\textwidth}
		\centering
		\includegraphics[scale=1]{Fig/honeywell_t-hawk.jpg} %1:1导入
	\end{minipage}%
	\begin{minipage}[c]{0.5\textwidth}
		\centering
		\includegraphics[scale=1]{Fig/GTSpy.jpg}
	\end{minipage}\\[1pt]
	\begin{minipage}[t]{0.5\textwidth}	% 以下为新添加页面划分,[t]表示行顶部对齐
		\caption{\label{Hawk}T-Hawk}
	\end{minipage}%
	\begin{minipage}[t]{0.5\textwidth}
		\caption{\label{GTSpy}GTSpy}
	\end{minipage}%
\end{figure}
\end{lstlisting}
\begin{figure}[htbp]
	\centering
	\begin{minipage}[c]{0.5\textwidth}
		\centering
		\includegraphics[width=6cm,height=6cm]{Fig/honeywell_t-hawk.jpg}
		\caption{\label{Hawk}T-Hawk}
	\end{minipage}%
	\begin{minipage}[c]{0.5\textwidth}
		\centering
		\includegraphics[width=6cm,height=6cm]{Fig/GTSpy.jpg}
		\caption{\label{GTSpy}GTSpy}
	\end{minipage}
\end{figure}


通常一个figure内含有其他小的figure,可以使用一些宏包,但最初本着简单的原则,本模板并没有使用这些子图包。后来应同学们要求在,把子图的功能加上,主要是修改了模板文件(scutthesis.cls文件)的功能包参数。注意,很多网上拿到的代码不一定可以精确的调子图标题字体字号,因为此模板的子图标题字体字号是利用subfig宏包的选项进行设置的(在scutthesis.cls文件的“图表环境”中),而有些教程使用subcaption进行同样的设置,还需进一步验证可行性。另外图的排版方法很多,有些宏包已经被弃用,所以尽量使用本文给出的案例的格式进行排版图片。

常见的子图包有subfigure和subfig。subfigure是比较老的了,这里使用subfig包。两个包在使用的时候用法不同,千万不要混淆了,不然可能会报错。subfig包的命令是\textbackslash{}subfloat。这里给出一种使用subfig包的常用排版,如图\ref{Fig:1}的子图\subref{Fig:1:b},其中\subref*{Fig:1:a}的试验并不好(这里测试了交叉引用\textbackslash{}subref\{xxx\}和\textbackslash{}subref*\{xxx\})。必要时也可以排版多行多列的图、调整图之间的间距,具体可百度。

\begin{lstlisting}
\begin{figure}[!h]
	\centering
	\subfloat[不合理的轨迹]{\includegraphics[width=6cm,height=6cm]{Fig/Figure_1.png}%
		\label{Fig:1:a}}
	\subfloat[优化的轨迹]{\includegraphics[width=6cm,height=6cm]{Fig/Figure_2.png}
		\label{Fig:1:b}}
	\\ % 用 \\ 换行,也可以此处空一行进行换行,只有两个图的话下面就不需要了。
	\subfloat[不合理的轨迹]{\includegraphics[width=6cm,height=6cm]{Fig/Figure_1.png}%
		\label{Fig:1:c}}
	\subfloat[优化的轨迹]{\includegraphics[width=6cm,height=6cm]{Fig/Figure_2.png}%
		\label{Fig:1:d}}
	\caption{子图包使用测试}\label{Fig:1}
\end{figure}
----------------------------------------------------------
% 引用某子图时使用\subref{xxx},其中xxx就是标签Fig:1:a
子图的引用比较特殊,命令有:\subref{xxx}和\subref*{xxx}
注:在subfig包使用说明中,\subref{xxx}和\subref*{xxx}分别由参数listofformat和subrefformat控制,
并由如下定义,根据撰写规范需要定义为:
\DeclareSubrefFormat{empty}{}
\DeclareSubrefFormat{simple}{#1#2}
\DeclareSubrefFormat{parens}{#1 #2)}
\DeclareSubrefFormat{subsimple}{#2}
\DeclareSubrefFormat{subparens}{ #2)}
和
\DeclareCaptionListOfFormat{empty}{}
\DeclareCaptionListOfFormat{simple}{#1#2}
\DeclareCaptionListOfFormat{parens}{#1 #2)}
\DeclareCaptionListOfFormat{subsimple}{#2}
\DeclareCaptionListOfFormat{subparens}{ #2)}
\end{lstlisting}
\begin{figure}[!h]
	\centering
	\subfloat[不合理的轨迹]{\includegraphics[width=6cm,height=6cm]{Fig/Figure_1.png}%
		\label{Fig:1:a}}
	\subfloat[优化的轨迹]{\includegraphics[width=6cm,height=6cm]{Fig/Figure_2.png}
 		\label{Fig:1:b}}
	\\ % 用 \\ 换行,也可以此处空一行进行换行
	\subfloat[不合理的轨迹]{\includegraphics[width=6cm,height=6cm]{Fig/Figure_1.png}%
		\label{Fig:1:c}}
	\subfloat[优化的轨迹]{\includegraphics[width=6cm,height=6cm]{Fig/Figure_2.png}%
		\label{Fig:1:d}}
	\caption{子图包使用测试}\label{Fig:1}
\end{figure}





需要局部更改字号时,可以使用tutorial文件夹lshort-zh-cn.pdf的5.1节进行更改,如加\textbackslash{}small使得字号小一号。
\section{定理}
在scutthesis.cls文件的最后,已经用\textbackslash{}newtheorem命令定义了几种定理环境,包括:定义、假设、定理、结论、引理、公理、推论、性质等等,统称定理环境,关于\textbackslash{}newtheorem的用法,可参考\cite{_g,_c}或自行百度。要下面提供几个例子,在横线之间的深色区域是代码,效果在相应下方表示:
\begin{lstlisting}
\begin{assumption}
	加权矩阵${{\boldsymbol{W}}_{1}}$和 ${{\boldsymbol{W}}_{2}}$ 是对称矩阵,且$ {{\boldsymbol{W}}_{2}}$非奇异。	\label{assum_dca1}
\end{assumption}
\end{lstlisting}
\begin{assumption}
	加权矩阵${{\boldsymbol{W}}_{1}}$和 ${{\boldsymbol{W}}_{2}}$ 是对称矩阵,且$ {{\boldsymbol{W}}_{2}}$非奇异。	\label{assum_dca1}
\end{assumption}

定理用法和假设类似:
\begin{lstlisting}
\begin{theorem}
	如果假设\ref{assum_dca1}成立,$\boldsymbol{F}$满足式\eqref{eq_F}的定义,且${{\boldsymbol{W}}_{1}}$非奇异,则有$0\le e \left( \boldsymbol{F} \right) < 1$,其中$e \left( \boldsymbol{F} \right)$是 $\boldsymbol{F}$的特征值。	\label{the_dca2}
\end{theorem}
\end{lstlisting}
\begin{theorem}
	如果假设\ref{assum_dca1}成立,$\boldsymbol{F}$满足上式的定义,且${{\boldsymbol{W}}_{1}}$非奇异,则有$0\le e \left( \boldsymbol{F} \right) < 1$,其中$e \left( \boldsymbol{F} \right)$是 $\boldsymbol{F}$的特征值。	\label{the_dca2}
\end{theorem}
\begin{remark}
	定理环境的编号可自定义,但通常不需要再进行设置,因为模板文件scutthesis.cls文件已经定义好。
\end{remark}

---------------------------------------------------------

2022年5月更新:

根据最新的博士论文送审结果,定理等环境统一把原来的斜体改成正体。在此引用一下参考文献\cite{_g}的内容:

amsthm 提供了 \textbackslash{}theoremstyle 命令支持定理格式的切换,在用 \textbackslash{}newtheorem 命令定义定 理环境之前使用。amsthm 预定义了三种格式用于 \textbackslash{}theoremstyle:plain 和 LATEX 原始的格式 一致;definition 使用粗体标签、正体内容;remark 使用斜体标签、正体内容。

以上部分在scutthesis.cls文件最后一部分设置。

---------------------------------------------------------

amsthm 还提供了一个 proof 环境用于排版定理的证明过程。proof 环境末尾自动加上一个证毕符号:
\begin{proof}
	显然有
	\[
	E=mc^2
	\]
	证毕
\end{proof}

proof的大更多用法见参考文献\cite{_g}。scutthesis.cls文件的最后,跟所有定理环境一样,只是把英文”Proof“改成中文“证明”。

\section{参考文献}

再次强调,使用其他参考文献管理软件的用户以及不使用任何软件的“裸奔”的用户不需要关注任何关于zetero的东西。
\begin{lstlisting}
	关于参考文献这块,很多同学有疑问。只有记住一点:不管用什么参考文献管理工具,最终目的是生成一个bib文件,bib文件里是特定格式的文献信息。bib文件当作文本打开,里面就是文献的元数据。
\end{lstlisting}

通常学位论文参考文献是基于BibTeX进行的,本模板使用的是BibLaTeX,或者叫Biber。关于这部分知识可参考文献\parencite{_c,_g}的第六章,6.1节参考文献和BIBTEX工具。所以使用TeXstudio或者vscode的时候需要注意调整正确的参数进行编译。

引用前手动加空格,如:

引用前没有加空格\parencite{_c,_g}的第六章,引用后面有空格。

引用前手动加空格 \parencite{_c,_g}的第六章,引用后面有空格。

手写方括号 [6]。引用后面没空格。

生成方括号 \parencite{_k}。引用后面没空格。


参考文献引用和著录是基于ZOTERO这个软件进行的。视频教程见\parencite{_k}。此外,为了符合毕业论文撰写规范,需设置参数。按照视频教程安装完必要的插件(如Better BibTeX)后,在编辑->首选项进行设置。图\ref{op1}到图\ref{op11}所示的是我的zotero软件设置。其中最重要的是\ref{op10}的设置要排除的选项,多余的显示会让审稿人反感,按照论文撰写规范进行即可。在毕业论文撰写时,在编辑->首选项->Better BibLTeX->Fields中,Fields to omit from export填month,abstract,note,extra,file,keywords,type,url,doi,就是在参考文献著录中排除这些多余的项,避免过于复杂。而在写本模板使用说明时,没有排除url,因为很多参考资料是网页。

\begin{lstlisting}
    使用zotero,有时候科学上网很重要。
\end{lstlisting}

在zotero软件点击文件->导出文献库,如图\ref{output}所示,再在导出对话框图\ref{output_format}选择导出格式为Better BibLaTeX,同时勾选Keep updated选项保持自动更新,再点击ok,在弹出的对话框图\ref{output_name}确定保存路径和文件名,例如我的是MyLibrary.bib,这也是我整个读书生涯的文献库bib文件。如果写小论文的话通常导出格式是BibTeX或者Better BibTeX(这里按照期刊的要求来即可,文献管理软件的好处就是快速自动生成一个文件库)。关于BibTeX和BibLaTeX的区别这里不做展开。

得到文献库后,在scutthesis.tex文件第九行使用\textbackslash{}addbibresource命令,添加文献库。引用某文献时秩序在zotero选中某文献条目,然后按Ctrl+Shift+C,复制引用关键字(Citation Key)到剪切板(快捷键可自定义)。然后在tex文件编辑界面直接粘贴,默认的时上标形式,若需要非上标形式,可以改为\textbackslash{}parencite\{xxx\},其中xxx是Citation Key。这里的操作和认为设置的首选项参数有关,需要在编辑->首选项->导出界面的默认格式一栏选中相应的项,同时在编辑->首选项->高级->快捷键设置为默认值。

---------------------------------------------------------------------------------

2020年12月2日测试:
下载最新zotero,从知网和谷歌捕获文献(刚打开网页最好稍等一会再点击插件,谷歌可能需要现人机验证),对文献\parencite{Renduchintala_2019}、\parencite{milz2020design}进行引用。

---------------------------------------------------------------------------------

---------------------------------------------------------------------------------

2021年9月14日测试:
使用endnote的用户也可以利用导出的bib文件生成参考文献著录信息,导出选项是bibTeX,貌似没有更多导出设置选项。导出设置没有zotero那么灵活丰富,得到bib文件后要引用某论文需要自行查找标签(label,也有软件叫引用关键字Citation Key)\{xxx\}然后手打\textbackslash{}cite\{xxx\}。欢迎熟悉endnote的同学来信告诉我更好的办法。

---------------------------------------------------------------------------------

2023年3月8日测试:参考文献管理软件经常更新,但还是那句话,无论什么工具,最终得到bib文件即可,在期刊的文章页或者谷歌学术搜索页,只需要复制/下载bibtex的内容。得到这些元数据后甚至自己往bib文件里加都可以。

---------------------------------------------------------------------------------

2023年11月测试。论文写完记得断掉bib文件自动更新,在zotero的插件Better BibTeX自动导出设置里删除不希望再继续同步到项。否则更改软件中的文献后,论文的bib文件也同步更改,但有时候这不是想要的。
\begin{lstlisting}
    另外有同学反映,换了电脑后重新导出的bib文件Citation Key值不同,记得设置好Better BibTeX之后,在著录条目界面全选著录(或仅选想更新的著录)然后右键选Better BibTeX更新refresh一下。然后在Automatic export选项点击Export now立即更新bib文件(按理说勾选了自动更新选项他会自动更新,但为了确保万无一失还是点一下)。
\end{lstlisting}
\begin{figure}
	\centering
	\includegraphics[scale=0.8]{Fig/zotero1.png}
	\caption{\label{op1}常规}
\end{figure}
\begin{figure}
	\centering
	\includegraphics[scale=0.8]{Fig/zotero2.png}
	\caption{\label{op2}同步1}
\end{figure}
\begin{figure}
	\centering
	\includegraphics[scale=0.8]{Fig/zotero3.png}
	\caption{\label{op3}同步2}
\end{figure}
\begin{figure}
	\centering
	\includegraphics[scale=0.8]{Fig/zotero4.png}
	\caption{\label{op4}搜索}
\end{figure}
\begin{figure}
	\centering
	\includegraphics[scale=0.8]{Fig/zotero5.png}
	\caption{\label{op5}导出}
\end{figure}
\begin{figure}
	\centering
	\includegraphics[scale=0.8]{Fig/zotero6.png}
	\caption{\label{op6}引用}
\end{figure}
\begin{figure}
	\centering
	\includegraphics[scale=0.8]{Fig/zotero7.png}
	\caption{\label{op7}高级1}
\end{figure}
\begin{figure}
	\centering
	\includegraphics[scale=0.8]{Fig/zotero8.png}
	\caption{\label{op8}高级2}
\end{figure}
\begin{figure}
	\centering
	\includegraphics[scale=0.8]{Fig/zotero9.png}
	\caption{\label{op9}Better BibTeX1}
\end{figure}
\begin{figure}
	\centering
	\includegraphics[scale=0.8]{Fig/zotero10.png}
	\caption{\label{op10}Better BibTeX2}
\end{figure}
\begin{figure}
	\centering
	\includegraphics[scale=0.8]{Fig/zotero11.png}
	\caption{\label{op11}Better BibTeX3}
\end{figure}

\begin{figure}[htbp]
	\centering
	\includegraphics[scale=0.42]{Fig/zotero12.png}
	\caption{\label{output}导出文献库}
\end{figure}

\begin{figure}[htbp]
	\centering
	\includegraphics[scale=0.42]{Fig/zotero13.png}
	\caption{\label{output_format}导出格式}
\end{figure}

\begin{figure}[htbp]
	\centering
	\includegraphics[scale=0.42]{Fig/zotero14.png}
	\caption{\label{output_name}导出文件名}
\end{figure}









