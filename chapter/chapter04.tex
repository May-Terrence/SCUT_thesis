\chapter{涵道风扇式无人机的位置控制}

为了实现对轨迹规划结果的准确跟踪,一个跟踪性能优良、抗扰能力强的位置控制器是必不可少的。本文构建了基于分层解耦的外环-内环模块化控制框架,外环控制即位置控制器,内环控制即姿态控制器。位置控制作为轨迹跟踪的核心决策单元,首先通过融合IMU、卫星定位模块和气压计等多源传感器数据,实时解算无人机在三维空间中的位置、速度等系统状态信息。然后需要根据当前时刻的位置误差、速度误差和受力状态等因素,计算出涵道风扇的总拉力(对应转速指令$\Omega$),并结合期望的姿态指令$\boldsymbol{\eta}_d=[\varphi_d\quad \theta_d\quad \psi_d]^T$,实现期望的位置、速度控制。基于解耦的模块化控制框架,实际的位置跟踪问题可以按照单独的外环控制来考虑,而不用考虑内环控制。并且模块化设计使得位置控制回路可以独立于姿态动力学环节进行控制器参数调节,有效地降低了DFUAV这一多变量耦合系统的设计复杂度。对于内环控制,则采用第三章介绍的基于INDI和优先级控制分配的姿态控制策略,期望的姿态指令由位置控制器解算得到。

在户外复杂气流环境中,阵风引起的非定常气流扰动是制约DFUAV位置控制精度的重要因素。计算流体力学仿真研究表明,当环境风速达到4.6m/s时,建筑物周围空气流速最高可以达到7.6m/s,这无疑会对DFUAV造成非常强烈的干扰,显著影响DFUAV的稳定性。因此,设计一个能够有效抗风的位置控制器尤为重要。针对该问题,广泛使用的PID位置控制器在有阵风的情况下存在其固有的缺陷:积分环节I在面对持续的风扰动时,补偿效果具有明显的相位滞后特性;而微分环节面对高频风速干扰则容易引发超调和震荡。如果环境风速变化已知,那么就可以及时补偿扰动以减小位置误差。为了测量环境风速,一种可能的解决方案是使用风速计测量机体速度与环境风速的相对速度,但环境风速可能来自各个方向,所以至少需要在无人机机体上安装三个风速计,这无疑会增加DFUAV硬件系统设计的复杂性和成本。并且风速计在低速的情况下存在明显噪声,误差较大。

第三章介绍了基于传感器的INDI方法的优越性:使用陀螺仪的测量值进行一阶差分来计算角加速度,进而基于角加速度的测量值估计出一个执行器周期内对未建模动态力矩的影响并实时补偿。牛顿第二定律揭示了加速度的大小与力的大小成正比关系$(\Delta \boldsymbol{F}\propto\Delta \boldsymbol{a})$,所以基于机载IMU中的加速度计三轴比力的测量值,可以直接估计出DFUAV所受环境风扰动的等效外力矢量。这一点与使用陀螺仪估计未建模动态力矩变化的思路是一致的。既然DFUAV的姿态可以被控制,那么位置控制同样可以采用这种思路,设计出一个用于控制DFUAV线性加速度的增量控制器。

因此本章安排如下:首先对位置动力学模型进行简化处理以适应INDI位置控制器的设计。然后根据加速度计可估算力这一优势,设计基于INDI的位置控制器。最后进行仿真实验与实际飞行实验,验证所提出的位置控制器的有效性。

\section{位置动力学模型简化}

在设计位置控制器之前,首先需要对位置动力学模型进行简化处理。根据本章开篇时的分析,可以使用IMU中的加速度计对无人机机体加速度的测量来估算机体所受外力。加速度计对加速度的测量本质上是基于比力的物理概念。在惯性导航系统中,比力被定义为无人机所受的合外力(包括气动力、推进力等)与重力矢量作向量差然后再除以飞行器的总质量。比力具有与加速度相同的量纲(m/s$^2$)。根据式\eqref{eq_7},$\boldsymbol{F}^b$表示的合外力已除去重力作用,所以比力在机体坐标系下可以表示为:
\begin{equation}
    \boldsymbol{A}^b=
    \begin{bmatrix}
    A_x^b \\
    A_y^b \\
    A_z^b
    \end{bmatrix}=\frac{1}{m}\boldsymbol{F}^b
    \label{4_1}
\end{equation}

DFUAV位置子系统的控制输入为涵道风扇的转速$\Omega$,为与加速度量纲统一,类似姿态控制,结合式\eqref{eq_22}引入具有加速度量纲的位置控制的虚拟控制输入$\boldsymbol{a}_u$:
\begin{equation}
    \begin{aligned}
        \boldsymbol{a}_u&=\frac{1}{m}\boldsymbol{R}_b^e\boldsymbol{F}^b_{fan}\\
        &=\frac{1}{m}\boldsymbol{R}_b^e\begin{bmatrix}0 \\ 0 \\
            -k_{fan}\Omega^2
        \end{bmatrix}
    \end{aligned}
    \label{4_2}
\end{equation}
因此只要求解出期望的$\boldsymbol{a}_u$即可根据无人机质量、旋转矩阵(姿态)和系数$k_{fan}$得到期望的涵道风扇转速$\Omega$。

由于位置平移运动在地面坐标系下分析较为方便,因此式\eqref{4_2}中左乘以旋转矩阵$\boldsymbol{R}_b^e$,可以在地面坐标系下表示三轴加速度输入。其中$\boldsymbol{a}_u$可描述为无人机姿态$\boldsymbol{\eta}$和风扇转速$\Omega$的函数:
\begin{equation}
    \boldsymbol{a}_u=\boldsymbol{a}_u(\boldsymbol{\eta},\Omega)
    \label{4_3}
\end{equation}

类似地,将不包括重力作用的合外力矢量$\boldsymbol{F}^b$除去涵道风扇的拉力矢量统一表示为DFUAV的气动力,使用$\bar{\boldsymbol{A}}^e$表示气动力在地面坐标系下的加速度:
\begin{equation}
    \bar{\boldsymbol{A}}^e=
    \begin{bmatrix}
    A_x^e \\
    A_y^e \\
    \bar{A}_z^e
    \end{bmatrix}=\boldsymbol{R}_b^e
    \begin{bmatrix}
    A_x^b \\
    A_y^b \\
    \bar{A}_z^b
    \end{bmatrix}=\frac{1}{m}\boldsymbol{R}_b^e(\boldsymbol{F}^b-\boldsymbol{F}^b_{fan})
    \label{4_4}
\end{equation}

根据建模分析中$\boldsymbol{F}^b$的组成可知,$\bar{\boldsymbol{A}}^e$可表示为姿态$\boldsymbol{\eta}$、速度$\boldsymbol{V}^e$和环境风速$\boldsymbol{W}^e$的函数:
\begin{equation}
    \bar{\boldsymbol{A}}^e=\bar{\boldsymbol{A}}^e(\boldsymbol{\eta},\boldsymbol{V}^e,\boldsymbol{W}^e)
    \label{4_5}
\end{equation}

结合式\eqref{eq_7}、\eqref{4_3}和\eqref{4_5},可以得到位置动力学模型的简化形式:
\begin{equation}
    \begin{aligned}
        \dot{\boldsymbol{V}}^e&=\dfrac{1}{m}\boldsymbol{R}_b^e\boldsymbol{F}^b+g\boldsymbol{e}_3\\
        &=\bar{\boldsymbol{A}}^e+\boldsymbol{a}_u+g\boldsymbol{e}_3\\
        &=  \begin{bmatrix}A_x^e \\A_y^e \\\bar{A}_z^e\end{bmatrix}+
        \begin{bmatrix}a_{ux} \\a_{uy} \\a_{uz}\end{bmatrix}+
        \begin{bmatrix}0 \\0 \\g\end{bmatrix}
    \end{aligned}
    \label{4_6}
\end{equation}

根据简化后的位置动力学模型\eqref{4_6},可将动力学分量分为两部分:第一部分为外环虚拟控制输入$\boldsymbol{a}_u$,这部分分量与实际的外环控制输入$\Omega$有关,用于控制外环状态(无人机的位置、速度),将通过下一小节的INDI位置控制器设计求解;第二部分为$\bar{\boldsymbol{A}}^e+g\boldsymbol{e}_3$,可以认为这部分分量是机体受到的扰动,会导致位置误差,将通过INDI位置控制器进行扰动补偿。

\section{基于INDI的位置控制器设计}

本节将INDI控制方法进一步应用于位置(外环)控制中,使用机载IMU中的加速度计来估计环境风速扰动,并通过INDI位置控制器根据扰动的大小实时补偿以提高DFUAV的抗风性能。这种外环-内环协同的扰动观测补偿机制具有显著的技术优势:一方面,加速度计对外力变化敏感,能够以1000Hz的频率检测环境风速扰动,相比于本章开篇时提到的风速计方案避免了安装复杂度和低速情况下的测量噪声问题;另一方面,内外环均采用基于传感器的扰动补偿策略,形成了从力矩扰动到力扰动的全链路扰动观测与补偿闭环控制框架,在阵风场景下可以抵消姿态震荡与位置偏移的耦合效应,可以有效地提高系统的稳定性和鲁棒性。

\subsection{INDI位置控制架构}

不同于姿态旋转运动\eqref{eq_42},位置平移运动\eqref{eq_40}不涉及三个位置通道的耦合,因此常采用如下的线性化位置误差:
\begin{equation}
    \begin{aligned}
        \boldsymbol{e}_P&=\boldsymbol{P}^e_d-\boldsymbol{P}^e\\
        &=\begin{bmatrix}x_d^e \\y_d^e \\z_d^e\end{bmatrix}-\begin{bmatrix}x^e \\y^e \\z^e\end{bmatrix}
    \end{aligned}
    \label{4_7}
\end{equation}

因此期望的速度指令可以表示为:
\begin{equation}
    \begin{aligned}
        \boldsymbol{V}_d^e&=\boldsymbol{K}_P\boldsymbol{e}_P+\frac{d\boldsymbol{P}^e_d}{dt}
    \end{aligned}
    \label{4_8}
\end{equation}
此处采用比例控制加前馈控制的方式,其中$\boldsymbol{K}_P=diag({K}_{Px},{K}_{Py},{K}_{Pz})$为位置误差的正定增益矩阵。

对于与动力学相关的速度动态,位置动力学模型简化中已经将其简化为以下形式:
\begin{equation}
    \begin{aligned}
        \dot{\boldsymbol{V}}^e&=
        \bar{\boldsymbol{A}}^e(\boldsymbol{\eta},\boldsymbol{V}^e,\boldsymbol{W}^e)+\boldsymbol{a}_u+g\boldsymbol{e}_3
    \end{aligned}
    \label{4_9}
\end{equation}
可以直接在式\eqref{4_9}的基础上应用INDI方法。根据第三章的INDI相关理论,为了得到$\boldsymbol{\dot{V}}^e$的增量表达式,对式\eqref{4_9}在工作点处作一阶泰勒展开:
\begin{equation}
    \begin{aligned}
        \boldsymbol{\dot{V}}^e&=\bar{\boldsymbol{A}}^e(\boldsymbol{\eta}_0,\boldsymbol{V}^e_0,\boldsymbol{W}^e_0)+\boldsymbol{a}_{u0}+g\boldsymbol{e}_3\\
        &+\left.\frac{\partial \bar{\boldsymbol{A}}^e}{\partial\boldsymbol{\eta}}+\right|_{\boldsymbol{\eta}=\boldsymbol{\eta}_0}
        (\boldsymbol{\eta}-\boldsymbol{\eta}_0)+\left.\frac{\partial \bar{\boldsymbol{A}}^e}{\partial\boldsymbol{V}^e}\right|_{\boldsymbol{V}^e=\boldsymbol{V}^e_0}
        (\boldsymbol{V}^e-\boldsymbol{V}^e_0)\\
        &+\left.\frac{\partial \bar{\boldsymbol{A}}^e}{\partial\boldsymbol{W}^e}\right|_{\boldsymbol{W}^e=\boldsymbol{W}^e_0}(\boldsymbol{W}^e-\boldsymbol{W}^e_0)+(\boldsymbol{a}_{u}-\boldsymbol{a}_{u0})
    \end{aligned}
    \label{4_10}
\end{equation}
在位置动力学模型简化小节中,定义了具有加速度量纲的$\boldsymbol{a}_u(\boldsymbol{\eta},\Omega)$,用来表示位置控制的虚拟控制输入,所以泰勒展开式中不对其细化求偏导数,而是将其看作一个整体来进行求解。

展开式中的第一项$\bar{\boldsymbol{A}}^e(\boldsymbol{\eta}_0,\boldsymbol{V}^e_0,\boldsymbol{W}^e_0)+\boldsymbol{a}_{u0}+g\boldsymbol{e}_3$等价于工作点处的加速度,这个值可以由机载IMU测量之后再转换到地面坐标系下并加上重力矢量得到,可以使用$\dot{\boldsymbol{V}}^e_0$表示:
\begin{equation}
        \dot{\boldsymbol{V}}^e_0=\bar{\boldsymbol{A}}^e(\boldsymbol{\eta}_0,\boldsymbol{V}^e_0,\boldsymbol{W}^e_0)+\boldsymbol{a}_{u0}+g\boldsymbol{e}_3
    \label{4_11}
\end{equation}

展开式中的其他项是关于系统内部状态变量$(\boldsymbol{V}^e,\boldsymbol{\eta})$、系统控制输入$\boldsymbol{a}_{u}$和外部环境风速扰动$\boldsymbol{W^e}$的一阶偏导数项。根据第三章的INDI理论介绍部分的时间尺度分离法则,可以认为在加速度的输入信号$\Omega$(对应虚拟控制输入$\boldsymbol{a}_u$)的时间尺度内,系统内部的状态变量的变化是缓慢的,包括展开式中的速度$\boldsymbol{V}^e$和姿态$\boldsymbol{\eta}$。因此可以将这些状态变量的一阶偏导数项近似视为0:
\begin{equation}
    \begin{cases}
        \boldsymbol{V}^e-\boldsymbol{V}_0^e\approx0 \\
        \boldsymbol{\eta}-\boldsymbol{\eta}_0\approx0 \\
    \end{cases}
    \label{4_12}
\end{equation}

此外,对于外部环境的风速扰动$\boldsymbol{W^e}$,类似姿态控制方案设计中的分析,最佳的假设同样是认为$(\boldsymbol{W}^e-\boldsymbol{W}_0^e\approx0)$。所有由风速变化导致的空气动力作用都已经包含在了$\dot{\boldsymbol{V}}^e_0$中。

其余关于系统控制输入项为$(\boldsymbol{a}_{u}-\boldsymbol{a}_{u0})$。因此可以得到速度动态的增量表达式:
\begin{equation}
    \begin{aligned}
        \boldsymbol{\dot{V}}^e&=\boldsymbol{\dot{V}}^e_0+\boldsymbol{a}_{u}-\boldsymbol{a}_{u0}\\
        &=(\boldsymbol{R}_b^e)_0\begin{bmatrix}A_{x0}^e \\A_{y0}^e \\A_{z0}^e\end{bmatrix}+
        \begin{bmatrix}0 \\0 \\g\end{bmatrix}+
        \begin{bmatrix}a_{ux} \\a_{uy} \\a_{uz}\end{bmatrix}-
        \begin{bmatrix}a_{ux0} \\a_{uy0} \\a_{uz0}\end{bmatrix}
    \end{aligned}
    \label{4_13}
\end{equation}

INDI通过对式\eqref{4_13}求逆,得到期望的位置控制虚拟控制输入的增量表达式: 
\begin{equation}
    \begin{aligned}
        \boldsymbol{a}_{ud}&=\boldsymbol{\kappa}-\boldsymbol{\dot{V}}^e_0+\boldsymbol{a}_{u0}\\
        &=\begin{bmatrix}\boldsymbol{\kappa}_x \\\boldsymbol{\kappa}_y \\\boldsymbol{\kappa}_z\end{bmatrix}-
        (\boldsymbol{R}_b^e)_0\begin{bmatrix}A_{x0}^e \\A_{y0}^e \\A_{z0}^e\end{bmatrix}-
        \begin{bmatrix}0 \\0 \\g\end{bmatrix}+
        \begin{bmatrix}a_{ux0} \\a_{uy0} \\a_{uz0}\end{bmatrix}
    \end{aligned}
    \label{4_14}
\end{equation}
其中,使用$\boldsymbol{a}_{ud}$替换$\boldsymbol{a}_{u}$表明是期望的虚拟控制输入。

将式\eqref{4_14}代入式\eqref{4_13}所示的增量表达式中,可以得到线性化的速度动态方程:
\begin{equation}
    \dot{\boldsymbol V}^e=\boldsymbol{\kappa}
    \label{4_15}
\end{equation}

对于式\eqref{4_15}所示的一阶积分系统,仅使用具有正反馈增益的比例控制器就可以确保系统的稳定性,为提高系统响应速度和精度,此处也可以加入前馈控制:
\begin{equation}
    \boldsymbol{\kappa}=\boldsymbol{K}_{V}(\boldsymbol{V}_d^e-\boldsymbol{V}^e)+\frac{d\boldsymbol{V}^e_d}{dt}
    \label{4_16}
\end{equation}
值得一提的是,前馈控制的加入不会影响系统传递函数的分母,即加入前后系统的极点不变,从而稳定性也不受影响。

将式\eqref{4_16}代入式\eqref{4_15}中可以得到:
\begin{equation}
    \dot{\boldsymbol V}^e=\boldsymbol{K}_{V}(\boldsymbol{V}_d^e-\boldsymbol{V}^e)+\frac{d\boldsymbol{V}^e_d}{dt}
    \label{4_17}
\end{equation}
其中$\boldsymbol{K}_V=diag({K}_{V x},{K}_{V y},{K}_{V z})$为正反馈增益矩阵。

将式\eqref{4_16}代入式\eqref{4_14}中可以得到:
\begin{equation}
    \begin{aligned}
        \boldsymbol{a}_{ud}&=\boldsymbol{K}_{V}(\boldsymbol{V}_d^e-\boldsymbol{V}^e)+\frac{d\boldsymbol{V}^e_d}{dt}-\boldsymbol{\dot{V}}^e_0+\boldsymbol{a}_{u0}\\
        &= \left[\begin{array}{ccc}{K}_{V x} & & \\& {K}_{V y} & \\& & {K}_{V z}\end{array}\right]
        \left (\begin{bmatrix}V^e_{dx} \\ V^e_{dy} \\ V^e_{dz} \end{bmatrix}-\begin{bmatrix}V^e_{x} \\ V^e_{y} \\ V^e_{z}
        \end{bmatrix}\right )+
        \begin{bmatrix}\frac{dV^e_{dx}}{dt} \\\frac{dV^e_{dy}}{dt} \\\frac{dV^e_{dz}}{dt}
        \end{bmatrix}-
        (\boldsymbol{R}_b^e)_0\begin{bmatrix}A_{x0}^e \\A_{y0}^e \\A_{z0}^e\end{bmatrix}-
        \begin{bmatrix}0 \\0 \\g\end{bmatrix}+
        \begin{bmatrix}a_{ux0} \\a_{uy0} \\a_{uz0}\end{bmatrix}
    \end{aligned}
    \label{4_18}
\end{equation}

文献\parencite{smeurCascadedIncrementalNonlinear2018b}中提供了另一种思路来寻找所需加速度输入的增量,该思路基于非线性方法。由于与动力学输入相关的非线性速度动态方程\eqref{4_9}是已知的,因此可以进行非线性反演。由于没有关于$\bar{\boldsymbol{A}}^e(\boldsymbol{\eta},\boldsymbol{V}^e,\boldsymbol{W}^e)$的精确估计,所以保持增量结构仍然是适合的。对方程\eqref{4_9}两边同时减去同一公式的此前最近一个工作点的表达式可以得到:
\begin{equation}
    \begin{aligned}
        \dot{\boldsymbol{V}}^e - \dot{\boldsymbol{V}}^e_0&=
        \bar{\boldsymbol{A}}^e(\boldsymbol{\eta},\boldsymbol{V}^e,\boldsymbol{W}^e) - \bar{\boldsymbol{A}}^e(\boldsymbol{\eta}_0,\boldsymbol{V}^e_0,\boldsymbol{W}^e_0) +\boldsymbol{a}_u - \boldsymbol{a}_{u0}+g\boldsymbol{e}_3 - g_0\boldsymbol{e}_3
    \end{aligned}
    \label{4_19}
\end{equation}

假设在这个小的时间瞬间内,$\bar{\boldsymbol{A}}^e(\boldsymbol{\eta},\boldsymbol{V}^e,\boldsymbol{W}^e)$和$g$的变化很小,所以有:
\begin{equation}
    \begin{aligned}
        \dot{\boldsymbol{V}}^e - \dot{\boldsymbol{V}}^e_0&=\boldsymbol{a}_u - \boldsymbol{a}_{u0}
    \end{aligned}
    \label{4_20}
\end{equation}
这个简化方程将虚拟控制输入的增量与加速度的增量联系起来,与式\eqref{4_13}的结果是一致的。

类似姿态控制,位置控制中的INDI在每一个工作点计算的虚拟控制输入$\boldsymbol{a}_{ud}$都是基于上一个工作点处的虚拟控制输入$\boldsymbol{a}_{u}$加上期望的加速度$\boldsymbol{\dot{V}}^e$再减去此刻通过加速度计测量值转换得到的加速度$\boldsymbol{\dot{V}}^e_0$。而$\boldsymbol{\dot{V}}^e_0$中已经包含了相邻两个工作点之间气动力对机体的影响,因此负面影响不会随着时间积累,而是在每一个工作点实时补偿,有效地提升了系统的抗风扰性能。

由于$\dot{\boldsymbol{V}}^e_0$的获取依赖于加速度计的测量值,而加速度计又对机体震动十分敏感,在实际应用中的常用的做法是在飞控板所处的上部电子仓与机身连接处安装减震垫以减小机体震动对加速度计的影响。此外,加速度计的测量值还需要进行滤波处理。第三章介绍的姿态控制器中用到的陀螺仪也需要滤波处理,假设内环使用滤波器$f_{in}$过滤,内环的推力增量为$\boldsymbol{T}-\boldsymbol{T}_{f_{in}}$,外环使用滤波器$f_{out}$过滤,外环的推力增量为$\boldsymbol{T}-\boldsymbol{T}_{f_{out}}$。为了使外环的推力增量可以传递到内环,两个滤波器的形式和参数都需要相同\cite{smeurCascadedIncrementalNonlinear2018b}:
\begin{equation}
        f_{in}=f_{out}
    \label{4_21}
\end{equation}
所以加速度计使用的滤波器也是二阶巴特沃斯低通滤波器,截止频率为30Hz。

此外,为了保证所有下标为0的项都位于同一工作点,实践中还需要对虚拟控制输入和姿态进行同步滤波。为与内环滤波统一,同样使用下标$(.)_s$表示经过滤波器滤波后的项。
\begin{equation}
        \boldsymbol{a}_{ud}=\boldsymbol{K}_{V}(\boldsymbol{V}_d^e-\boldsymbol{V}^e)+\frac{d\boldsymbol{V}^e_d}{dt}-\boldsymbol{\dot{V}}^e_f+\boldsymbol{a}_{uf}
    \label{4_22}
\end{equation}

除了滤波外,加速度计测量的原始值往往有固定偏置,加速度计偏置的存在意味着DFUAV在静止状态下会有一个非零的加速度测量值,这种系统性误差会导致惯性导航结算中的速度与位置出现积分漂移,增加误差。为了消除这个偏置,需要在使用之前对加速度计进行校准。一般情况下采用六面校准法即可,若对精度要求高,则可采用高精度转台进行动态校准。

\subsection{期望姿态角与拉力解算}

期望的加速度输入$\boldsymbol{a}_{ud}$是在地面坐标系下的,而DFUAV的拉力向量始终作用在机体坐标系的$\boldsymbol{O}_b-\boldsymbol{Z}_b$轴方向上。根据式\eqref{4_2}以及旋转矩阵不影响矢量大小的性质,可以得到期望的涵道风扇拉力大小$\|\boldsymbol{F}_{fan}\|$:
\begin{equation}
    \begin{aligned}
        \|\boldsymbol{F}_{fan}\|&=m\|\boldsymbol{a}_{ud}\|\\
        &=m\sqrt{a_{udx}^2+a_{udy}^2+a_{udz}^2}
    \end{aligned}
    \label{4_23}
\end{equation}

根据拉力大小可求出涵道风扇的转速$\Omega$:
\begin{equation}
    \begin{aligned}
        \Omega&=\sqrt{\frac{\|\boldsymbol{F}_{fan}\|}{k_{fan}}}
    \end{aligned}
    \label{4_24}
\end{equation}





