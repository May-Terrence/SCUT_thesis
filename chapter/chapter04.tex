\chapter{涵道风扇式无人机的位置控制}

为了实现轨迹规划的目标,一个跟踪性能优良的位置控制器是必不可少的。本文构建了基于分层解耦的外环-内环模块化控制框架,外环控制即位置控制器,内环控制即姿态控制器。位置控制作为轨迹跟踪的核心决策单元,首先通过融合IMU、卫星定位模块和气压计等多源传感器数据,实时解算无人机在三维空间中的位置、速度等系统状态信息。然后需要根据当前时刻的位置误差、速度误差和受力状态等因素,计算出涵道风扇的总拉力(对应转速指令$\Omega$),并结合期望的姿态指令$\boldsymbol{\eta}_d=[\varphi_d\quad \theta_d\quad \psi_d]^T$,实现期望的位置控制。基于模块化控制框架,实际的位置跟踪问题可以按照单独的外环控制来考虑,而不用考虑内环控制。并且模块化设计使得位置控制回路可以独立于姿态动力学进行参数调节,有效的降低了DFUAV这一多变量耦合系统的设计复杂度。对于内环控制,则采用第三章介绍的基于INDI和优先级控制分配的姿态控制策略,期望的姿态指令由位置控制器解算得到。

在户外复杂气流环境中,阵风引起的非定常气流扰动是制约DFUAV位置控制精度的重要因素。计算流体力学仿真研究表明,当环境风速达到4.6m/s时,建筑物周围空气流速最高可达7.6m/s,这无疑会对DFUAV造成强烈的干扰,显著影响DFUAV的稳定性。因此,设计一个能够抗风的位置控制器尤为重要。针对该问题,广泛使用的PID位置控制器在有阵风的情况下存在固有缺陷:其积分环节在面对持续的风扰动时,补偿效果具有明显的相位滞后特性;而微分环节面对高频风速干扰则容易引发超调和震荡。如果风速变化已知,那么就可以很好的补偿扰动减小位置误差。为了测量风速,一种可能的解决方案是使用风速计测量机体与环境风速的相对速度,但环境风速可能来自各个方向,所以至少需要在无人机机体上安装三个风速计,这无疑会增加系统设计的复杂性、质量和成本。并且风速计在低速的情况下存在明显噪声,误差较大。

第三章介绍了基于传感器的INDI方法的优越性:使用陀螺仪的测量值进行一阶差分来计算角加速度,进而基于角加速度的测量值估计出一个执行器周期内未建模动态的影响和未知力矩扰动的影响并实时补偿。由于加速度与力成正比关系,所以机载加速度计数值的变化能够很好地反映出DFUAV所受外力的变化。这一点与使用陀螺仪估计力矩变化的思路是一致的。因此,本章将INDI控制方法进一步应用于外环控制中,使用加速度计来估计环境风速扰动以提高DFUAV的抗风性能。

本章安排如下:
\section{引言}
































现先总结出所推荐的间距设置,无编号的:
\begin{lstlisting}
\begin{itemize}[topsep = 0 pt, itemsep= 0 pt, parsep=0pt, partopsep=0pt, leftmargin=36pt, itemindent=0pt, labelsep=6pt, listparindent=24pt]
	\item 第一项。内容内容内容内容内容内容内容内容内容内容内容内容内容内容内容内容内容内容内容内容内容内容内容内容内容内容内容内容内容内容

	\item 第二项。内容内容内容内容内容内容内容内容内容内容内容内容内容内容内容内容内容内容内容内容内容内容内容内容内容内容内容内容内容内容
	
\end{itemize}
\end{lstlisting}
效果:
\begin{itemize}[topsep = 0 pt, itemsep= 0 pt, parsep=0pt, partopsep=0pt, leftmargin=36pt, itemindent=0pt, labelsep=6pt, listparindent=24pt]
	\item 第一项。内容内容内容内容内容内容内容内容内容内容内容内容内容内容内容内容内容内容内容内容内容内容内容内容内容内容内容内容内容内容

	\item 第二项。内容内容内容内容内容内容内容内容内容内容内容内容内容内容内容内容内容内容内容内容内容内容内容内容内容内容内容内容内容内容
	
\end{itemize}

有编号的:
\begin{lstlisting}
\begin{enumerate}[topsep = 0 pt, itemsep= 0 pt, parsep=0pt, partopsep=0pt, leftmargin=44pt, itemindent=0pt, labelsep=6pt, label=(\arabic*)]
	\item 第一项。内容内容内容内容内容内容内容内容内容内容内容内容内容内容内容内容内容内容内容内容内容内容内容内容内容内容内容内容内容内容

	\item 第二项。内容内容内容内容内容内容内容内容内容内容内容内容内容内容内容内容内容内容内容内容内容内容内容内容内容内容内容内容内容内容
	
\end{enumerate}
\end{lstlisting}
效果:
\begin{enumerate}[topsep = 0 pt, itemsep= 0 pt, parsep=0pt, partopsep=0pt, leftmargin=44pt, itemindent=0pt, labelsep=6pt, label=(\arabic*)]
	\item 第一项。内容内容内容内容内容内容内容内容内容内容内容内容内容内容内容内容内容内容内容内容内容内容内容内容内容内容内容内容内容内容

	\item 第二项。内容内容内容内容内容内容内容内容内容内容内容内容内容内容内容内容内容内容内容内容内容内容内容内容内容内容内容内容内容内容
	
\end{enumerate}

下面两节分别讨论参数设置规则。
\subsection{垂直间距}
摘抄宏包说明:
\begin{itemize}[topsep = 0 pt, itemsep= 0 pt, parsep=0pt, partopsep=0pt, leftmargin=36pt, itemindent=0pt, labelsep=6pt, listparindent=24pt]
	\item topsep控制列表环境与上文之间的距离。第一项和前一段之间的空间。

	\item itemsep 条目之间的距离

	\item parsep 条目里面段落之间的距离
 
	\item partopsep 条目与下面段落的距离。当环境开始一个新段落时,额外的空间被添加到 \textbackslash{}topsep。
\end{itemize}

论文中希望上述距离都为0pt,如:
\begin{lstlisting}
	\begin{itemize}[topsep = 0 pt, itemsep= 0 pt, parsep=0pt, partopsep=0pt]
		\item 第一项。
		\item 第二项
		\item 第三项。
	\end{itemize}
\end{lstlisting}
效果为:
\begin{itemize}[topsep = 0 pt, itemsep= 0 pt, parsep=0pt, partopsep=0pt]
	\item 第一项。
	\item 第二项
	\item 第三项。
\end{itemize}


\subsection{水平间距}
% 正文12pt
水平间距调整比较复杂,对照宏包说明给出的图,下面内容参考了宏包原文和网络资料:
\begin{itemize}[topsep = 0 pt, itemsep= 0 pt, parsep=0pt, partopsep=0pt, leftmargin=36pt, itemindent=0pt, labelsep=6pt, listparindent=24pt]
	\item 为页面的左边距)和该列表的左边距之间的空间。 必须是非负数。 它的值取决于表,则为页面的左边距)和该列表的左边距之间的空间。 必须是非负数。 它的值取决于列表级别。
	\item rightmargin       列表环境右边的空白长度。类似于 \textbackslash{}leftmargin 但用于右边距。 它的值通常是 0pt。
	\item labelsep       标号与列表第一项文本左侧的距离。标签框的末尾和第一项的文本之间的空间。 它的默认值为 0.5 em。
	\item itemindent       条目的缩进距离。添加到项目第一行文本部分的水平缩进的额外缩进。 通过减去 labelsep 和 labelwidth 的值,相对于该参考点计算标签的起始位置。 它的值通常是 0pt。注:理解这个变量时,查看图\ref{enumitem}的顺序应该按照箭头从左到右,先leftmargin再itemindent,然后再labelsep,最后labelwidth。即箭头的起始点是基准点。若itemindent=0pt,则leftmargin-labelsep-编号长度的结果就是编号起始位置。
	\item labelwidth       包含标签的框的标称宽度。 如果标签的自然宽度为 < labelwidth,则默认情况下,标签在宽度为 (labelwidth) 的框内右对齐排版。否则,使用自然宽度的框,这会导致该行上的文本缩进。 可以通过为 \textbackslash{}makelabel 命令提供定义来修改标签的排版方式。
	\item listparindent       条目下面段落的缩进距离。除了以 litem 开头的段落之外,列表的每个段落的开头都有额外的缩进。 可以为负数,但通常为 0pt。
\end{itemize}

无编号的水平间距,给出两张方案
 
第一种:
\begin{itemize}[topsep = 0 pt, itemsep= 0 pt, parsep=0pt, partopsep=0pt, leftmargin=36pt, itemindent=0pt, labelsep=6pt, listparindent=24pt]
	\item 第一项。内容内容内容内容内容内容内容内容内容内容内容内容内容内容内容内容内容内容内容内容内容内容内容内容内容内容内容内容内容内容

	% 第一项的第二段。内容内容内容内容内容内容内容内容内容内容内容内容内容内容内容内容内容内容内容内容内容内容内容内容内容内容内容内容内容内容
	\item 第二项。内容内容内容内容内容内容内容内容内容内容内容内容内容内容内容内容内容内容内容内容内容内容内容内容内容内容内容内容内容内容
	
\end{itemize}


第二种:
\begin{itemize}[topsep = 0 pt, itemsep= 0 pt, parsep=0pt, partopsep=0pt, leftmargin=0pt, itemindent=36pt, labelsep=6pt, listparindent=24pt]
	\item 第一项。内容内容内容内容内容内容内容内容内容内容内容内容内容内容内容内容内容内容内容内容内容内容内容内容内容内容内容内容内容内容

	% 第一项的第二段。内容内容内容内容内容内容内容内容内容内容内容内容内容内容内容内容内容内容内容内容内容内容内容内容内容内容内容内容内容内容
	\item 第二项。内容内容内容内容内容内容内容内容内容内容内容内容内容内容内容内容内容内容内容内容内容内容内容内容内容内容内容内容内容内容
	
\end{itemize}

推荐第一种。

有编号的水平间距,下面给出三种方案:
注:labelsep是某一项文字和编号框的距离,一般就设为一个空格6pt,要使编号左侧缩进两格,itemindent-labelsep要等于编号长度。注意编号是右对齐,向左扩展的。

第一种方案是整体右移两格,文字距离编号一个空格,然后第二行文字和第一行对齐:
\begin{enumerate}[topsep = 0 pt, itemsep= 0 pt, parsep=0pt, partopsep=0pt, leftmargin=44pt, itemindent=0pt, labelsep=6pt, label=(\arabic*)]
	\item 第一项。内容内容内容内容内容内容内容内容内容内容内容内容内容内容内容内容内容内容内容内容内容内容内容内容内容内容内容内容内容内容

	% 第一项的第二段。内容内容内容内容内容内容内容内容内容内容内容内容内容内容内容内容内容内容内容内容内容内容内容内容内容内容内容内容内容内容
	\item 第二项。内容内容内容内容内容内容内容内容内容内容内容内容内容内容内容内容内容内容内容内容内容内容内容内容内容内容内容内容内容内容
	
\end{enumerate}

第二种方案是和论文撰写规范的格式一样,注意不是论文撰写规范规定的格式,规范里没有规定这些格式。如:
\begin{enumerate}[topsep = 0 pt, itemsep= 0 pt, parsep=0pt, partopsep=0pt, leftmargin=0pt, itemindent=44pt, labelsep=6pt, listparindent=24pt, label=(\arabic*)]
	\item 第一项。内容内容内容内容内容内容内容内容内容内容内容内容内容内容内容内容内容内容内容内容内容内容内容内容内容内容内容内容内容内容

	% 第一项的第二段。内容内容内容内容内容内容内容内容内容内容内容内容内容内容内容内容内容内容内容内容内容内容内容内容内容内容内容内容内容内容
	\item 第二项。内容内容内容内容内容内容内容内容内容内容内容内容内容内容内容内容内容内容内容内容内容内容内容内容内容内容内容内容内容内容
	
\end{enumerate}

第三种方案是整体右移两格,文字距离编号一个空格,第二行文字不再右移:
\begin{enumerate}[topsep = 0 pt, itemsep= 0 pt, parsep=0pt, partopsep=0pt, leftmargin=24pt, itemindent=20pt, labelsep=6pt, listparindent=20pt, label=(\arabic*)]
	\item 第一项。内容内容内容内容内容内容内容内容内容内容内容内容内容内容内容内容内容内容内容内容内容内容内容内容内容内容内容内容内容内容

	% 第一项的第二段。内容内容内容内容内容内容内容内容内容内容内容内容内容内容内容内容内容内容内容内容内容内容内容内容内容内容内容内容内容内容
	\item 第二项。内容内容内容内容内容内容内容内容内容内容内容内容内容内容内容内容内容内容内容内容内容内容内容内容内容内容内容内容内容内容
	
\end{enumerate}

推荐第一种。

\section{enumerate标签样式}
除上述小括号数字的编号方法外,还有斜体字母等。在使用enumerate的时候,label的问题就是使用计数的字符,是阿拉伯数字、罗马、中文、还是希腊字符的问题。

\subsection{小括号阿拉伯数字}
% 小括号阿拉伯数字,用 label=\arabic*)
\begin{enumerate}[topsep = 0 pt, itemsep= 0 pt, parsep=0pt, partopsep=0pt, leftmargin=0pt, itemindent=44pt, labelsep=6pt, listparindent=24pt, label=\arabic*)]
	\item 第一项。

	% 第一项的第二段。
	\item 第二项
	
	% 第二项的第二段。
	\item 第三项。
	
	% 第三项的第二段。
\end{enumerate}



\subsection{斜体字母}
% 斜体字母,用 label=\emph{\alph*}
\begin{enumerate}[topsep = 0 pt, itemsep= 0 pt, parsep=0pt, partopsep=0pt, leftmargin=0pt, itemindent=44pt, labelsep=6pt, listparindent=24pt, label=\emph{\alph*}.]
	\item 第一项。

	% 第一项的第二段。
	\item 第二项
	
	% 第二项的第二段。
	\item 第三项。
	
	% 第三项的第二段。
\end{enumerate}

\subsection{大写罗马字母}
% 大写罗马字母,用 label=(\Roman*)
\begin{enumerate}[topsep = 0 pt, itemsep= 0 pt, parsep=0pt, partopsep=0pt, leftmargin=0pt, itemindent=44pt, labelsep=6pt, listparindent=24pt, label=(\Roman*)]
	\item 第一项。

	% 第一项的第二段。
	\item 第二项
	
	% 第二项的第二段。
	\item 第三项。
	
	% 第三项的第二段。
\end{enumerate}