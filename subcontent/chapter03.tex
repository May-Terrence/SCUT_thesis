
\chapter{常用环境及参考文献设置}
强烈建议在使用公式、表格、定理环境时进行百度,没必要研究各种用法,只需要知道自己需要什么。因本人的论文所用表格较少,因而对表格不是很熟悉,本章对表格的介绍相应的较少。本章仅介绍本人在论文撰写过程中常用的环境以及参考文献设置。

\section{图}
图的导入需要提前准备好图片文件,最好是.png、.eps、.pdf或.jpg文件。另外,如果是从matlab导出图片文件,可使用print函数或手动导出,print函数的使用可参考ICGNC2020plot.m以及PlotToFileColorPDF.m文件等。手动导出(matlab的figure界面的“文件”->“导出设置”设置好大小、分辨率和线宽等然后点击“应用于图窗”)主要用于观察效果,可设置某种样式名称后保存该样式,下次使用时加载,具体可百度“matlab导出高清图片”。需要特别注意的是一定要1:1导入matlab生成的图片,并且图中文字设置好字体字号。否则缩放之后,图片的字号就变了,盲审老师一眼就能看出来字号不对,就很麻烦。这就是为什么要在matlab点击“应用于图窗“进行预览,观测效果后再1:1使用图片。

使用如下代码放置独立成行的图片,效果如图\ref{one_DFUAV}所示
\begin{lstlisting}
\begin{figure}[htbp]
	% 图片居中(列居中对齐)
	\centering	
	% 包含当前路径下的Fig文件夹的图片文件DFUAV_f31.png
	\includegraphics[scale=1]{Fig/DFUAV_f31.png} 
	% 添加标签one_DFUAV以及图标题“涵道风扇式无人机”,引用某图时使用\ref{xxx},其中xxx就是标签,图编号是自动生成的。
	\caption{\label{one_DFUAV}涵道风扇式无人机} 
\end{figure}
\end{lstlisting}
其中figure为环境名,[htbp]表示将图片设置为浮动体,实际上这在.cls文件已经设置过,因而可以省略。[scale=1]表示安装1:1的比例导入图片,还可以按其他方式导入,需要时可自行百度。
\begin{figure}[htbp]
	\centering
	\includegraphics[scale=1]{Fig/DFUAV_f31.png}
	\caption{\label{one_DFUAV}涵道风扇式无人机}
\end{figure}

使用如下代码划分页面并排放置图\ref{Hawk}、图\ref{GTSpy}
\begin{lstlisting}
\begin{figure}[htbp]
	\centering
	\begin{minipage}[c]{0.5\textwidth} % minipage将页面划分为0.5\textwidth
		\centering
		\includegraphics[width=6cm,height=6cm]{Fig/honeywell_t-hawk.jpg}
		\caption{\label{Hawk}T-Hawk}
	\end{minipage}%
	\begin{minipage}[c]{0.5\textwidth}
		\centering
		\includegraphics[width=6cm,height=6cm]{Fig/GTSpy.jpg}
		\caption{\label{GTSpy}GTSpy}
	\end{minipage}
\end{figure}
\end{lstlisting}
其中[c]表示行居中对齐。当图片大小不一但又需要1:1导入时,图标题可能行不对齐,因此可以改为如下指令:
\begin{lstlisting}
\begin{figure}[htbp]
	\centering
	\begin{minipage}[c]{0.5\textwidth}
		\centering
		\includegraphics[scale=1]{Fig/honeywell_t-hawk.jpg} %1:1导入
	\end{minipage}%
	\begin{minipage}[c]{0.5\textwidth}
		\centering
		\includegraphics[scale=1]{Fig/GTSpy.jpg}
	\end{minipage}\\[1pt]
	\begin{minipage}[t]{0.5\textwidth}	% 以下为新添加页面划分,[t]表示行顶部对齐
		\caption{\label{Hawk}T-Hawk}
	\end{minipage}%
	\begin{minipage}[t]{0.5\textwidth}
		\caption{\label{GTSpy}GTSpy}
	\end{minipage}%
\end{figure}
\end{lstlisting}
\begin{figure}[htbp]
	\centering
	\begin{minipage}[c]{0.5\textwidth}
		\centering
		\includegraphics[width=6cm,height=6cm]{Fig/honeywell_t-hawk.jpg}
		\caption{\label{Hawk}T-Hawk}
	\end{minipage}%
	\begin{minipage}[c]{0.5\textwidth}
		\centering
		\includegraphics[width=6cm,height=6cm]{Fig/GTSpy.jpg}
		\caption{\label{GTSpy}GTSpy}
	\end{minipage}
\end{figure}


通常一个figure内含有其他小的figure,可以使用一些宏包,但最初本着简单的原则,本模板并没有使用这些子图包。后来应同学们要求在,把子图的功能加上,主要是修改了模板文件(scutthesis.cls文件)的功能包参数。注意,很多网上拿到的代码不一定可以精确的调子图标题字体字号,因为此模板的子图标题字体字号是利用subfig宏包的选项进行设置的(在scutthesis.cls文件的“图表环境”中),而有些教程使用subcaption进行同样的设置,还需进一步验证可行性。另外图的排版方法很多,有些宏包已经被弃用,所以尽量使用本文给出的案例的格式进行排版图片。

常见的子图包有subfigure和subfig。subfigure是比较老的了,这里使用subfig包。两个包在使用的时候用法不同,千万不要混淆了,不然可能会报错。subfig包的命令是\textbackslash{}subfloat。这里给出一种使用subfig包的常用排版,如图\ref{Fig:1}的子图\subref{Fig:1:b},其中\subref*{Fig:1:a}的试验并不好(这里测试了交叉引用\textbackslash{}subref\{xxx\}和\textbackslash{}subref*\{xxx\})。必要时也可以排版多行多列的图、调整图之间的间距,具体可百度。

\begin{lstlisting}
\begin{figure}[!h]
	\centering
	\subfloat[不合理的轨迹]{\includegraphics[width=6cm,height=6cm]{Fig/Figure_1.png}%
		\label{Fig:1:a}}
	\subfloat[优化的轨迹]{\includegraphics[width=6cm,height=6cm]{Fig/Figure_2.png}
		\label{Fig:1:b}}
	\\ % 用 \\ 换行,也可以此处空一行进行换行,只有两个图的话下面就不需要了。
	\subfloat[不合理的轨迹]{\includegraphics[width=6cm,height=6cm]{Fig/Figure_1.png}%
		\label{Fig:1:c}}
	\subfloat[优化的轨迹]{\includegraphics[width=6cm,height=6cm]{Fig/Figure_2.png}%
		\label{Fig:1:d}}
	\caption{子图包使用测试}\label{Fig:1}
\end{figure}
----------------------------------------------------------
% 引用某子图时使用\subref{xxx},其中xxx就是标签Fig:1:a
子图的引用比较特殊,命令有:\subref{xxx}和\subref*{xxx}
注:在subfig包使用说明中,\subref{xxx}和\subref*{xxx}分别由参数listofformat和subrefformat控制,
并由如下定义,根据撰写规范需要定义为:
\DeclareSubrefFormat{empty}{}
\DeclareSubrefFormat{simple}{#1#2}
\DeclareSubrefFormat{parens}{#1 #2)}
\DeclareSubrefFormat{subsimple}{#2}
\DeclareSubrefFormat{subparens}{ #2)}
和
\DeclareCaptionListOfFormat{empty}{}
\DeclareCaptionListOfFormat{simple}{#1#2}
\DeclareCaptionListOfFormat{parens}{#1 #2)}
\DeclareCaptionListOfFormat{subsimple}{#2}
\DeclareCaptionListOfFormat{subparens}{ #2)}
\end{lstlisting}
\begin{figure}[!h]
	\centering
	\subfloat[不合理的轨迹]{\includegraphics[width=6cm,height=6cm]{Fig/Figure_1.png}%
		\label{Fig:1:a}}
	\subfloat[优化的轨迹]{\includegraphics[width=6cm,height=6cm]{Fig/Figure_2.png}
 		\label{Fig:1:b}}
	\\ % 用 \\ 换行,也可以此处空一行进行换行
	\subfloat[不合理的轨迹]{\includegraphics[width=6cm,height=6cm]{Fig/Figure_1.png}%
		\label{Fig:1:c}}
	\subfloat[优化的轨迹]{\includegraphics[width=6cm,height=6cm]{Fig/Figure_2.png}%
		\label{Fig:1:d}}
	\caption{子图包使用测试}\label{Fig:1}
\end{figure}

\section{表}
本节仅展示使用常见的三线表
\begin{lstlisting}
\begin{table}
	\caption{\label{TDF_para}涵道模型参数}	%表题在上
	\centering	% 表居中
	\small	% 表内字体小一号(即设置成和表题字号一致)
	\begin{tabular}{cccc}	% cccc表示4列并居中,若列之间需要分隔符则设置为|c|c|c|c|
		\hline	% \hline表示横线。列之间的元素用&分隔,\tabularnewline表示换行
		参数符号 & 数值 & 参数符号 & 数值 \tabularnewline 
		\hline 
		$I_x$ & $054593$ 		   & $I_y$ & $0.017045         $ \tabularnewline
		$l_1$ & $0.0808\,\text{m}$ & $l_2$ & $0.175\,\text{m}  $ \tabularnewline 
		$l_4$ & $0.2415\,\text{m}$ & $l_5$ & $0.1085\,\text{m} $ \tabularnewline
		\hline 
	\end{tabular}
\end{table}
\end{lstlisting}
\begin{table}
	\caption{\label{TDF_para}涵道模型参数}
	\centering
	\small 
	\begin{tabular}{cccc}
		\hline 
		参数符号 & 数值                & 参数符号 & 数值                 \tabularnewline
		\hline 
		$I_x$   & $054593$ 		     & $I_y$   & $0.017045         $ \tabularnewline
		$l_1$   & $0.0808\,\text{m}$ & $l_2$   & $0.175\,\text{m}  $ \tabularnewline 
		$l_4$   & $0.2415\,\text{m}$ & $l_5$   & $0.1085\,\text{m} $ \tabularnewline
		\hline 
	\end{tabular}
\end{table}

\section{公式}
除了前面讲行内公式,常用的还有行间公式。公式中的数学符号可自行百度,本章仅介绍常用的几种公式环境。

单独成行的行间公式在 \LaTeX{} 里由equation 环境包裹。equation 环境为公式自动生成一个编号,这个编号可以用\textbackslash{}label 和\textbackslash{}ref 生成交叉引用,amsmath 宏包的\textbackslash{}eqref 可为引用自动加上圆括号;如式\eqref{eq_1}所示。
\begin{lstlisting}
\begin{equation}
	a+b=c	\label{eq_1}
\end{equation}
\end{lstlisting}
\begin{equation}
	a+b=c	\label{eq_1}
\end{equation}
若不需要编号则加星号,改为
\begin{lstlisting}
\begin{equation*}
	a+b=c
\end{equation*}
\end{lstlisting}
其他环境类似。当使用 \texttt\$ 开启行内公式输入,或是使用{equation} 环境时,\LaTeX\ 就进入了数学模式。
数学模式相比于文本模式有以下特点:
\begin{enumerate}
	\item 数学模式中输入的空格被忽略。数学符号的间距默认由符号的性质(关系符号、运算符等)决定。
	需要人为引入间距时,使用 \textbackslash{}{quad} 和 \textbackslash{}{qquad} 等命令。
	\item {不允许有空行(分段)}。行间公式中也无法用 $ \verb|\\|$命令手动换行。排版多行公式需要用到 其他各种环境。
	\item 所有的字母被当作数学公式中的变量处理,字母间距与文本模式不一致,也无法生成单词之间的空格。
	如果想在数学公式中输入正体的文本,简单情况下可用 \textbackslash{}{mathrm} 命令。
	或者用 {amsmath} 提供的 \textbackslash{}{text} 命令(仅适合在公式中穿插少量文字。如果你的情况正好相反,需要在许多文字中穿插使用公式,则应该像正常的行内公式那样用,而不是滥用 \textbackslash{}{text} 命令)。
\end{enumerate}	

实际上更常用的的是多行公式,不需要对齐的公式组可以使用gather环境,需要对齐的公式组用align 环境。
长公式内可用$ \verb|\\|$ 换行。

如果需要罗列一系列公式,并令其按照等号对齐,可用align 环境,它将公式用\& 隔为两部分并对齐。分隔符通常放在等号左边:
\begin{lstlisting}
\begin{align}
	a & = b + c \\
	& = d + e
\end{align}
\end{lstlisting}
\begin{align}
a & = b + c \\
& = d + e
\end{align}
align 环境会给每行公式都编号。

如果不需要按等号对齐,只需罗列数个公式,可用gather环境:
\begin{lstlisting}
\begin{gather}
	a  = b + c \notag \\
	f = d + e 
\end{gather}
\end{lstlisting}
\begin{gather}
	a  = b + c \notag  \\
	f = d + e 
\end{gather}
gather 环境同样会给每行公式都编号,如果某行不需要编号可在行末用\textbackslash{}notag 仅去掉某行的编号。

align 和gather 有对应的不带编号的版本align* 和gather*。

另一个常见的需求是将多个公式组在一起公用一个编号,编号位于公式的居中位置。为此,
amsmath 宏包提供了诸如aligned、gathered 等环境,与equation 环境套用。以-ed 结尾的
环境用法与前一节不以-ed 结尾的环境用法一一对应。我们仅以aligned 举例:
\begin{lstlisting}
\begin{equation}
	\begin{aligned}
		a &= b + c \\
		d &= e + f + g \\
		h + i &= j + k \\
		l + m &= n
	\end{aligned}
\end{equation}
\end{lstlisting}
\begin{equation}
	\begin{aligned}
		a &= b + c \\
		d &= e + f + g \\
		h + i &= j + k \\
		l + m &= n
	\end{aligned}
\end{equation}
split 环境和aligned 环境用法类似,也用于和equation 环境套用,区别是split 只能
将每行的一个公式分两栏,aligned 允许每行多个公式多栏。

分段函数通常用amsmath 宏包提供的cases 环境,可参考文献\parencite{_c}

amsmath 宏包还直接提供了多种排版矩阵的环境,包括不带定界符的matrix,以及带各种定界符的矩阵pmatrix、bmatrix、Bmatrix、vmatrix、Vmatrix。
其中中括号版的bmatrix最常用。这些矩阵环境需要在公式中使用,比如 gather 环境。
\begin{lstlisting}
\begin{gather}
	\boldsymbol{A}= \begin{bmatrix}
		x_{11} & x_{12} & \ldots & x_{1n} \\
		x_{21} & x_{22} & \ldots & x_{2n} \\
		\vdots & \vdots & \ddots & \vdots \\
		x_{n1} & x_{n2} & \ldots & x_{nn}
	\end{bmatrix}
\end{gather}
\end{lstlisting}
\begin{gather}
\boldsymbol{A}= \begin{bmatrix}
	x_{11} & x_{12} & \ldots & x_{1n} \\
	x_{21} & x_{22} & \ldots & x_{2n} \\
	\vdots & \vdots & \ddots & \vdots \\
	x_{n1} & x_{n2} & \ldots & x_{nn}
   \end{bmatrix}
\end{gather}	
其中矩阵/向量加粗使用\textbackslash{}boldsymbol\{\}命令,\textbackslash{}bm\{\}命令和unicode-math包有兼容性问题。另外还可以使用array环境排版矩阵,类似tabular环境,用$ \verb|\\|$ 和\& 用来分隔行和列,这里不再赘述。	
\begin{lstlisting}
\begin{array }[外部对齐tcb]{列对齐lcr}
	行列内容
\end{array}
\end{lstlisting}

另外注意排版分式时,有两种方法:\textbackslash{}frac或者\textbackslash{}dfrac,效果分别为$ \frac{1}{2} $和$ \dfrac{1}{2} $。以上介绍的数学环境中,空格可参考文献\parencite{_c},例如常用\textbackslash{}quad。

需要局部更改字号时,可以使用tutorial文件夹lshort-zh-cn.pdf的5.1节进行更改,如加\textbackslash{}small使得字号小一号。
\section{定理}
在scutthesis.cls文件的最后,已经用\textbackslash{}newtheorem命令定义了几种定理环境,包括:定义、假设、定理、结论、引理、公理、推论、性质等等,统称定理环境,关于\textbackslash{}newtheorem的用法,可参考\cite{_g,_c}或自行百度。要下面提供几个例子,在横线之间的深色区域是代码,效果在相应下方表示:
\begin{lstlisting}
\begin{assumption}
	加权矩阵${{\boldsymbol{W}}_{1}}$和 ${{\boldsymbol{W}}_{2}}$ 是对称矩阵,且$ {{\boldsymbol{W}}_{2}}$非奇异。	\label{assum_dca1}
\end{assumption}
\end{lstlisting}
\begin{assumption}
	加权矩阵${{\boldsymbol{W}}_{1}}$和 ${{\boldsymbol{W}}_{2}}$ 是对称矩阵,且$ {{\boldsymbol{W}}_{2}}$非奇异。	\label{assum_dca1}
\end{assumption}

定理用法和假设类似:
\begin{lstlisting}
\begin{theorem}
	如果假设\ref{assum_dca1}成立,$\boldsymbol{F}$满足式\eqref{eq_F}的定义,且${{\boldsymbol{W}}_{1}}$非奇异,则有$0\le e \left( \boldsymbol{F} \right) < 1$,其中$e \left( \boldsymbol{F} \right)$是 $\boldsymbol{F}$的特征值。	\label{the_dca2}
\end{theorem}
\end{lstlisting}
\begin{theorem}
	如果假设\ref{assum_dca1}成立,$\boldsymbol{F}$满足上式的定义,且${{\boldsymbol{W}}_{1}}$非奇异,则有$0\le e \left( \boldsymbol{F} \right) < 1$,其中$e \left( \boldsymbol{F} \right)$是 $\boldsymbol{F}$的特征值。	\label{the_dca2}
\end{theorem}
\begin{remark}
	定理环境的编号可自定义,但通常不需要再进行设置,因为模板文件scutthesis.cls文件已经定义好。
\end{remark}

---------------------------------------------------------

2022年5月更新:

根据最新的博士论文送审结果,定理等环境统一把原来的斜体改成正体。在此引用一下参考文献\cite{_g}的内容:

amsthm 提供了 \textbackslash{}theoremstyle 命令支持定理格式的切换,在用 \textbackslash{}newtheorem 命令定义定 理环境之前使用。amsthm 预定义了三种格式用于 \textbackslash{}theoremstyle:plain 和 LATEX 原始的格式 一致;definition 使用粗体标签、正体内容;remark 使用斜体标签、正体内容。

以上部分在scutthesis.cls文件最后一部分设置。

---------------------------------------------------------

amsthm 还提供了一个 proof 环境用于排版定理的证明过程。proof 环境末尾自动加上一个证毕符号:
\begin{proof}
	显然有
	\[
	E=mc^2
	\]
	证毕
\end{proof}

proof的大更多用法见参考文献\cite{_g}。scutthesis.cls文件的最后,跟所有定理环境一样,只是把英文”Proof“改成中文“证明”。

\section{参考文献}

再次强调,使用其他参考文献管理软件的用户以及不使用任何软件的“裸奔”的用户不需要关注任何关于zetero的东西。
\begin{lstlisting}
	关于参考文献这块,很多同学有疑问。只有记住一点:不管用什么参考文献管理工具,最终目的是生成一个bib文件,bib文件里是特定格式的文献信息。bib文件当作文本打开,里面就是文献的元数据。
\end{lstlisting}

通常学位论文参考文献是基于BibTeX进行的,本模板使用的是BibLaTeX,或者叫Biber。关于这部分知识可参考文献\parencite{_c,_g}的第六章,6.1节参考文献和BIBTEX工具。所以使用TeXstudio或者vscode的时候需要注意调整正确的参数进行编译。

引用前手动加空格,如:

引用前没有加空格\parencite{_c,_g}的第六章,引用后面有空格。

引用前手动加空格 \parencite{_c,_g}的第六章,引用后面有空格。

手写方括号 [6]。引用后面没空格。

生成方括号 \parencite{_k}。引用后面没空格。


参考文献引用和著录是基于ZOTERO这个软件进行的。视频教程见\parencite{_k}。此外,为了符合毕业论文撰写规范,需设置参数。按照视频教程安装完必要的插件(如Better BibTeX)后,在编辑->首选项进行设置。图\ref{op1}到图\ref{op11}所示的是我的zotero软件设置。其中最重要的是\ref{op10}的设置要排除的选项,多余的显示会让审稿人反感,按照论文撰写规范进行即可。在毕业论文撰写时,在编辑->首选项->Better BibLTeX->Fields中,Fields to omit from export填month,abstract,note,extra,file,keywords,type,url,doi,就是在参考文献著录中排除这些多余的项,避免过于复杂。而在写本模板使用说明时,没有排除url,因为很多参考资料是网页。

\begin{lstlisting}
    使用zotero,有时候科学上网很重要。
\end{lstlisting}

在zotero软件点击文件->导出文献库,如图\ref{output}所示,再在导出对话框图\ref{output_format}选择导出格式为Better BibLaTeX,同时勾选Keep updated选项保持自动更新,再点击ok,在弹出的对话框图\ref{output_name}确定保存路径和文件名,例如我的是MyLibrary.bib,这也是我整个读书生涯的文献库bib文件。如果写小论文的话通常导出格式是BibTeX或者Better BibTeX(这里按照期刊的要求来即可,文献管理软件的好处就是快速自动生成一个文件库)。关于BibTeX和BibLaTeX的区别这里不做展开。

得到文献库后,在scutthesis.tex文件第九行使用\textbackslash{}addbibresource命令,添加文献库。引用某文献时秩序在zotero选中某文献条目,然后按Ctrl+Shift+C,复制引用关键字(Citation Key)到剪切板(快捷键可自定义)。然后在tex文件编辑界面直接粘贴,默认的时上标形式,若需要非上标形式,可以改为\textbackslash{}parencite\{xxx\},其中xxx是Citation Key。这里的操作和认为设置的首选项参数有关,需要在编辑->首选项->导出界面的默认格式一栏选中相应的项,同时在编辑->首选项->高级->快捷键设置为默认值。

---------------------------------------------------------------------------------

2020年12月2日测试:
下载最新zotero,从知网和谷歌捕获文献(刚打开网页最好稍等一会再点击插件,谷歌可能需要现人机验证),对文献\parencite{Renduchintala_2019}、\parencite{milz2020design}进行引用。

---------------------------------------------------------------------------------

---------------------------------------------------------------------------------

2021年9月14日测试:
使用endnote的用户也可以利用导出的bib文件生成参考文献著录信息,导出选项是bibTeX,貌似没有更多导出设置选项。导出设置没有zotero那么灵活丰富,得到bib文件后要引用某论文需要自行查找标签(label,也有软件叫引用关键字Citation Key)\{xxx\}然后手打\textbackslash{}cite\{xxx\}。欢迎熟悉endnote的同学来信告诉我更好的办法。

---------------------------------------------------------------------------------

2023年3月8日测试:参考文献管理软件经常更新,但还是那句话,无论什么工具,最终得到bib文件即可,在期刊的文章页或者谷歌学术搜索页,只需要复制/下载bibtex的内容。得到这些元数据后甚至自己往bib文件里加都可以。

---------------------------------------------------------------------------------

2023年11月测试。论文写完记得断掉bib文件自动更新,在zotero的插件Better BibTeX自动导出设置里删除不希望再继续同步到项。否则更改软件中的文献后,论文的bib文件也同步更改,但有时候这不是想要的。
\begin{lstlisting}
    另外有同学反映,换了电脑后重新导出的bib文件Citation Key值不同,记得设置好Better BibTeX之后,在著录条目界面全选著录(或仅选想更新的著录)然后右键选Better BibTeX更新refresh一下。然后在Automatic export选项点击Export now立即更新bib文件(按理说勾选了自动更新选项他会自动更新,但为了确保万无一失还是点一下)。
\end{lstlisting}
\begin{figure}
	\centering
	\includegraphics[scale=0.8]{Fig/zotero1.png}
	\caption{\label{op1}常规}
\end{figure}
\begin{figure}
	\centering
	\includegraphics[scale=0.8]{Fig/zotero2.png}
	\caption{\label{op2}同步1}
\end{figure}
\begin{figure}
	\centering
	\includegraphics[scale=0.8]{Fig/zotero3.png}
	\caption{\label{op3}同步2}
\end{figure}
\begin{figure}
	\centering
	\includegraphics[scale=0.8]{Fig/zotero4.png}
	\caption{\label{op4}搜索}
\end{figure}
\begin{figure}
	\centering
	\includegraphics[scale=0.8]{Fig/zotero5.png}
	\caption{\label{op5}导出}
\end{figure}
\begin{figure}
	\centering
	\includegraphics[scale=0.8]{Fig/zotero6.png}
	\caption{\label{op6}引用}
\end{figure}
\begin{figure}
	\centering
	\includegraphics[scale=0.8]{Fig/zotero7.png}
	\caption{\label{op7}高级1}
\end{figure}
\begin{figure}
	\centering
	\includegraphics[scale=0.8]{Fig/zotero8.png}
	\caption{\label{op8}高级2}
\end{figure}
\begin{figure}
	\centering
	\includegraphics[scale=0.8]{Fig/zotero9.png}
	\caption{\label{op9}Better BibTeX1}
\end{figure}
\begin{figure}
	\centering
	\includegraphics[scale=0.8]{Fig/zotero10.png}
	\caption{\label{op10}Better BibTeX2}
\end{figure}
\begin{figure}
	\centering
	\includegraphics[scale=0.8]{Fig/zotero11.png}
	\caption{\label{op11}Better BibTeX3}
\end{figure}

\begin{figure}[htbp]
	\centering
	\includegraphics[scale=0.42]{Fig/zotero12.png}
	\caption{\label{output}导出文献库}
\end{figure}

\begin{figure}[htbp]
	\centering
	\includegraphics[scale=0.42]{Fig/zotero13.png}
	\caption{\label{output_format}导出格式}
\end{figure}

\begin{figure}[htbp]
	\centering
	\includegraphics[scale=0.42]{Fig/zotero14.png}
	\caption{\label{output_name}导出文件名}
\end{figure}









